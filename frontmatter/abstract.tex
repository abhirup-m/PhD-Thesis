%! TEX root = thesis.tex
\chapter*{Abstract}

%%%%% COME BACK TO THIS, HOLOGRAPHY NOT DISCUSSED AT ALL
\inpar{A New Approach Towards Studying Correlated Electrons}This thesis reports the development of a new auxiliary model method for studying translation-invariant models of strongly-interacting electrons. The method allows us to create a mapping between a lattice model and an auxiliary impurity model in terms of the Hamiltonian, eigenstates, Greens functions and entanglement measures, among other things. This mapping ensures that such lattice model quantities can be computed indirectly from similar computations within the auxiliary impurity model. In terms of phenomenology, this mapping connects various renormalisation group (RG) fixed points of the impurity model to those of the lattice model. As examples, local Fermi liquid and local moment phases of the impurity site get mapped to Fermi liquid and Mott insulating phases, respectively, on the lattice model. The mapping makes use of translation operators and a manybody version of Bloch's theorem to recreate the lattice model via repeated ``tiling" operations on the impurity model. In contrast to other auxiliary model methods such as dynamical mean-field theory (DMFT) and its cluster variants (C-DMFT), this approach does not involve a self-consistency loop, making it more transparent. It also provides significantly more $k-$space resolution which is necessary for studying phenomena such as the pseudogap observed in the copper-oxide high-\(T_c\) superconductors. 

A short description of the protocol for applying our method is as follows. We first identify the appropriate quantum impurity model corresponding to the lattice model at hand. For example, the impurity model must have a localisation-delocalisation transition on the impurity site in order to capture a metal-insulator transition on the lattice model. As another example, topological features of the band structure can be encoded by working with a topologically non-trivial conduction bath within the impurity model. Having decided upon a quantum impurity model, we solve the model using an impurity solver. In the works described in this thesis, we use the unitary renormalisation group because of its (i) analytical transparency, (ii) ability to deal with weak and strong correlations, (iii) ability to work at zero temperature, and (iv) reduced computational cost. Using the URG, we obtain a hierarchy of Hamiltonians that describe the high-energy and low-energy behaviour of the auxiliary model in various parameter regimes. This requires numerically solving the URG flow equations for various Hamiltonian couplings.

The phases of the impurity model are then characterise by (a) effective Hamiltonians, (b) static equal-time correlations, (c) dynamical correlations such as Greens functions and self-energy, and (d) entanglement measures. Calculating some of these objects requires carrying out an iterative diagonalisation procedure on the hierarchy of Hamiltonians. Our tiling method then ``periodises" these objects into the lattice model, generating required $k-$dependence in the process. Since the conduction bath of the impurity model already inherits the lattice geometry of the parent model, the appropriate symmetries are baked into the quantities. As an example, the lattice $k-$space Greens function \(G_k\) obtains contributions from the impurity site Greens function as well as impurity-bath off-diagonal Greens functions. While the former captures the local component of \(G_k\), the latter captures the non-local components. With similar mappings existing for the dynamical correlations and entanglement measures, our method allows us to characterise various phases of the lattice model purely through impurity model computations. 

\inpar{Some Important Questions}In this thesis, we have used the auxiliary model tiling method to study three fundamental problems in quantum condensed matter physics: (i) Mott metal-insulator transition on the Bethe lattice in infinite dimensions, (ii) Mott metal-insulator transition on the 2D square lattice, and (ii) quantum criticality in heavy fermion materials in 2D. Towards this, we have applied the method to two paradigmatic models: (a) an extended Hubbard model, and (ii) a bilayer extended Hubbard model. In preparation for applying the tiling approach, we have first analysed the corresponding quantum impurity models.
\begin{itemize}
	\item[1.] The Mott transition on the Bethe lattice is studied through an extended Anderson impurity model (eSIAM) that hosts local electronic correlation in the conduction bath proximate to the impurity site. Due to the lack of non-local correlations in this limit, it is sufficient to treat the conduction electron in the continuum limit.
	\item[2.] The Mott transition in 2D is highly sensitive to details of the lattice. This requires us to study a lattice-embedded impurity model, where the impurity site is embedded in a 2D lattice of conduction electrons, and its hybridisation with the bath has the symmetries of the underlying lattice.
	\item[3.] The heavy fermions problem necessitates a two layer impurity model, each layer acting as a lattice-embedded impurity model and coupled via inter-layer hybridisation. 
\end{itemize}
The tiling approach then translates results from the auxiliary models into appropriate lattice models.
%%%% COME BACK TO THIS, NEED TO COMMENT ON THE HOLOGRAPHY PROJECT AND THE HEAVY FERMIONS PROJECT
In this thesis, we have attempted to answer a number of important open questions in the field:
\begin{itemize}
	\item {\it Is there a minimal but effective quantum impurity model Hamiltonian that describes the Mott MIT of the $1/2-$filled Hubbard model on the Bethe lattice in infinite dimensions?} Finding such an impurity model would also reduce the considerable computational effort that is presently required in self-consistent approaches. In the same context, is it possible to obtain a low-energy theory for the local gapless excitations precisely at the transition, where the metal is on the brink of destruction?
	\item For strongly-correlated electronic models, {\it is it possible to map the renormalisation group fixed points of such lattice models to those of quantum impurity models?} This would provide a firmer theoretical backing to various quantum impurity models, as well as reduce computational costs. 
	\item The phenomenon of Mott insulation involves the localization of itinerant electrons due to strong local repulsion. Upon doping, a pseudogap (PG) phase emerges - marked by selective gapping of the Fermi surface without conventional symmetry breaking in spin or charge channels. {\it A key challenge is understanding how quasiparticle breakdown in the Fermi liquid gives rise to this enigmatic state, and how it connects to both the Mott insulating and superconducting phases.}
\end{itemize}


\inpar{Summary Of The Chapters.}For the readers' convenience, we now briefly describe the content of the various chapters. 

%%%%% COME BACK TO THIS 
The \hyperref[sec:Introduction]{\it Introduction} ...

%%%%% COME BACK TO THIS, SOME MORE DETAILS CAN PROBABLY BE ADDED
After the introduction, the {\it Methods and Preliminaries} chapter describes the various techniques and some preliminary ideas employed throughout the thesis.
The chapter begins by deriving the unitary renormalisation group formalism and describes it's various properties and a practical protocol for implementing it. That is followed by demonstrations of the unitary RG on three simple models - a star graph model of spins in a magnetic field, the two-channel Kondo model, and the problem of a particle in a periodic potential. Through these problems, various aspects of the method are clarified.
In order to provide more insights, the chapter also details how the unitary RG method is related to other RG-based methods that rely on canonical transformations, such as Poor Man's scaling, Schrieffer-Wolff transformation and CUT (continuous unitary transformation) RG.
Apart from the URG, the chapter also describes an iterative diagonalisation method that has been used within this thesis to compute various correlation functions and entanglement measures from the hierarchy of Hamiltonians obtained from the URG.
Since various entanglement measures have been used throughout the later chapters, we have also added a section on various entanglement measures here - what they mean, what are their inter-relations, why they are important and how one can compute some of them.
The chapter concludes by describing some of the other strategies that have been employed to reduce computational costs in various projects.

In Chapter 3, {\it Holographic entanglement renormalisation for fermionic quantum matter}, we demonstrate the emergence of a holographic dimension in a system of 2D non-interacting Dirac fermions placed on a torus, by studying the scaling of multipartite entanglement measures under a sequence of renormalisation group (RG) transformations applied in momentum space.
Geometric measures defined in this emergent space can be related to the RG beta function of the spectral gap, hence establishing a holographic connection between the spatial geometry of the emergent spatial dimension and the entanglement properties of the boundary quantum theory.
We prove, analytically, that changing the boundedness of the holographic space involves a topological transition accompanied by a critical Fermi surface in the boundary theory.
We go on to show that this results in the formation of a quantum wormhole geometry that connects the UV and the IR of the emergent dimension.
The additional conformal symmetry at the transition also supports a relation between the emergent metric and the stress-energy tensor.
In the presence of an Aharonov–Bohm flux, the entanglement gains a geometry-independent piece which is shown to be topological, sensitive to changes in boundary conditions, and related to the Luttinger volume of the system.
Upon the insertion of a strong transverse magnetic field, we show that the Luttinger volume is linked to the Chern number of the occupied single-particle Landau levels.

In Chapter 4, {\it Kondo frustration via charge fluctuations: A route to Mott localisation}, we study our first impurity model, the extended Anderson model with local correlation \(U_b\) in the conduction bath. We demonstrate that this model hosts a local version of the Mott metal-insulator transition on the Bethe lattice in infinite dimensions, as seen from DMFT. At a critical value of the ratio of \(U_b\) and the impurity-bath hybridisation, the effective impurity model shows a transition from a Kondo screened phase into an unscreened local moment phase. For the case of attractive local bath correlations, the model sheds new light on several aspects of the DMFT phase diagram. For example, the T = 0 metal-to-insulator quantum phase transition (QPT) is preceded by an excited state QPT (ESQPT) where the local moment eigenstates are emergent in the low-lying spectrum. Long-ranged fluctuations are observed near both the QPT and ESQPT, suggesting that they are the origin of the quantum critical scaling observed recently at high temperatures in DMFT simulations. The $T = 0$ gapless excitations at the quantum critical point display particle-hole interconversion processes, and exhibit power-law behaviour in self-energies and two-particle correlations. These are signatures of non-Fermi liquid behaviour that emerge from the partial breakdown of the Kondo screening. A many-body perturbation theoretic treatment of the Hubbard sidebands reveals that they are comprised of holon-doublon excitations created by the hybridisation of the impurity site with the conduction bath. These excitations consist of decoupled local Fermi liquids for the holons and doublons at the lowest order, which, at higher orders, become coupled via correlated holon-doublon scattering between impurity and bath.

In Chapter 5, {\it Mapping From Auxiliary Model To Lattice: The Tiling Approach}, we lay out the details of the auxiliary model method developed to map impurity models to lattice models. We first motivate the general idea behind auxiliary model methods. This is followed by a construction of the manybody translation operators. We use this, along with a manybody Bloch's theorem, to construct lattice models from quantum impurity models and relate their eigenstates. This then allows us to map correlations functions, self-energies and entanglement measures across models. In particular, we demonstrate how the presence of the conduction bath introduces $k-$space resolution in various quantities.

In Chapter 6, {\it Mott Criticality as the Confinement Transition of a Pseudogap-Mott Metal: Insights from a lattice-embedded impurity model}, we apply the auxiliary model tiling method to a lattice-embedded extended SIAM. Applying a many-body tiling scheme to the fixed-point impurity model uncovers a lattice model with electron interactions and Kondo physics. At half-filling, the interplay between Kondo screening and bath charge fluctuations in the impurity model leads to Fermi liquid breakdown. This reveals a pseudogap phase characterized by a non-Fermi liquid (the Mott metal) residing on nodal arcs, gapped antinodal regions of the Fermi surface (Luttinger surfaces), and an anomalous scaling of the electronic scattering rate with frequency. The eventual confinement of holon-doublon excitations of this exotic metal obtains a continuous transition into the Mott insulator. Our results identify the pseudogap as a distinct long-range entangled quantum phase, and offer a new route to Mott criticality beyond the paradigm of local quantum criticality. The phase appears to encode multipartite entanglement, as seen from a large value of the quantum Fisher information. The critical point at the pseudogap-Mott insulator boundary is found to be described by the exactly-solvable Hatsugai-Kohmoto model. The emergence of Luttinger surfaces alters the anomaly structure associated with the generalized symmetry of the FS [44–46]. The corresponding topological charge is defined via the Luttinger–Ward functional. This charge is modified by Green’s function zeros, signalling a shift in the underlying anomaly. Within our framework, this reorganization grants topological protection to the RG flows terminating in the PG regime, ensuring the stability of its NFL excitations. 

%%%%%% COME BACK TO THIS In Chapter 7, {\it Quantum Criticality in Heavy Fermions},

We conclude in the final chapter by summarising the main results obtained within the thesis, in terms of the methods developed as well as insights obtained into the behaviour of various models. We describe the impact of these results on the field, the broad conclusions emerging from them and some open questions resulting from the work.
