\thesisTitle{WRITE YOUR THESIS TITLE HERE IN ENGLISH AND IN ALL CAPITAL LETTERS}
\thesisTitleTR{TEZ BAŞLIĞINIZIN TÜRKÇE ÇEVİRİSİNİ TÜM HARFLER BÜYÜK OLARAK  BURAYA YAZINIZ}
%\thesisYear{2018}
%\thesisMonth{August}\thesisMonthTR{Ağustos}
\thesisDefenseDay{25}  % day of defense, 1, 2, ..., 31
\thesisDegree{PhD}  %% PhD || MSc 
\thesisDegreeTR{Doktora} %% Doktora || Yüksek Lisans
\thesisProgram{Mechanical Engineering}
\thesisProgramTR{Makine Mühendisliği}
\thesisAuthor{Name SURNAME} % thesis author Name SURNAME
\thesisAdvisorTitle{Prof. Dr.}   % Assoc. Prof. Dr. , Assist. Prof. Dr.
\thesisAdvisorTitleTR{Prof. Dr.} % Doç. Dr.         , Dr. Öğr. Üyesi
\thesisAdvisor{Name SURNAME} % thesis advisor Name SURNAME (without title!)
\thesisAdvisorDepartment{Department of Mechanical Engineering}
%\thesisCoAdvisor{Assoc. Prof. Dr. Name SURNAME} % If there is co-advisor open this line
%\thesisCoAdvisorInstitution{BAUN} % If there is co-advisor open (un-comment) this line and write institution

\thesisChairman{Prof. Dr. ()}
\thesisMemberTWO{Prof. Dr. ()}
\thesisMemberTHREE{Prof. Dr. ()}
\thesisMemberFOUR{Prof. Dr. (KBÜ)}    % for MsC thesis with 3 committe member close (comment) this line
\thesisMemberFIVE{Assoc Prof. Dr. (KBÜ)} % for MsC thesis with 3 committe member close (comment) this line
%\thesisMemberSIX{Assoc. Prof. Dr. Name SURNAME (DEU)} % for PhD thesis with 7 committe member open (un-comment) this line and fill
%\thesisMemberSEVEN{Prof.Prof.Dr. Name SURNAME (IYTE)} % for PhD thesis with 7 committe member open (un-comment) this line and fill

\thesisPages{131} % The number of pages is not automatically counted! Enter it here at the end.

% write your thesis abstract within the curly braces
\thesisAbstract{ 
Nowadays, rapid consumption of fossil fuels and increasing electricity requirement have attracted attention on renewable energy resources. Today, thermoelectric (TE) power generation has become a promising technology in energy saving and reduction of environmental impacts in the world. Thermoelectric generators (TEGs), based on Seebeck effect, are used in power generation by recovering waste heat released from automobiles, factories, and similar resources.

In this study, high temperature thermoelectric generators with 8 pairs of rectangular prism ($\mathrm{TEG_{RP}}$) and cylindrical legs ($\mathrm{TEG_C}$) were fabricated individually and experimentally investigated. Within this scope, oxide thermoelectric materials of dually doped Ca$_{2.5}$Ag$_{0.3}$Eu$_{0.2}$Co$_4$O$_9$ and Zn$_{0.96}$Al$_{0.02}$Ga$_{0.02}$O with the highest figure of merit ($zT$) values of $0.57$ and $0.17$ were selected among 19 different compositions. TE powders were synthesized using sol-gel method following cold pressing for consolidation of $n-$ and $p-$type legs of both TEGs. Prior to the fabrication of the TEGs, geometrical optimization of the legs was performed for both TEGs using Response Surface Methodology and dimensions of the legs were specified.

Power generation characteristics of the TEGs were evaluated by establishing a measurement system and $I-V$ and $I-P$ curves were obtained. According to the results, open circuit voltage ($V_{OC}$) and maximum generated output power ($P_{max}$) of both TEGs increased with increasing $\Delta T$. Maximum $V_{OC}$ of $\mathrm{TEG_{RP}}$ and $\mathrm{TEG_C}$ were obtained as $133.1$~mV and $158.2$~mV, respectively, for the temperature difference ($\Delta T$) of $440\,^\circ$C at 495~$^\circ$C hot side temperature ($T_H$). As the current increased, output power was generated with respect to $P \propto I^2$. $P_{max}$ for $\mathrm{TEG_{RP}}$ and $\mathrm{TEG_C}$ were obtained as $33.7$~mW and 45.5~mW at the same conditions.

Steady-state thermal and thermal-electric analyses were performed to evaluate power generation performances and temperature distributions of $\mathrm{TEG_{RP}}$ and $\mathrm{TEG_C}$. $V_{OC}$ and $P_{max}$ of both TEGs increased with increasing $\Delta T$ as well as in the experimental results. Maximum $V_{OC}$ for both TEGs were obtained as 1117~mV for 410~$^\circ$C $\Delta T$ and 495~$^\circ$C $T_H$. $I-P$ curves for both TEGs exhibited similar characteristics with the experimental results. As a result, $P_{max}$ for $\mathrm{TEG_{RP}}$ and $\mathrm{TEG_C}$ were obtained as 3315.1~mW and 2632.9~mW in the same conditions. When the results of thermal-electric analyses are compared to the experimental results of fabricated TEGs, measured open circuit voltages and generated output powers were much lower than the results of thermal-electric analyses. In this study, noteworthy voltage and output power losses were correlated to the contact resistances between the TE legs and the silver conductors.
}
% write thesis key words separating by comma
\thesisKeywords{Thermoelectrics, oxide materials, material characterization, thermoelectric properties, thermoelectric generator, response surface optimization, thermal-electric analysis, power generation.}

% tezinizin özetinin Türkçe çevirisini süslü parantezler içerisine yazınız
\thesisOzet{Günümüzde fosil yakıtların hızla tüketilmesi ve elektriğe olan artış, dikkatleri yenilenebilir enerji kaynakları üzerine çekmektedir. Bugün termoelektrik (TE) güç üretimi, enerji tasarrufu ve dünyadaki çevresel etkilerin azaltılması noktasında umut verici bir teknoloji haline gelmiştir. Temeli Seebeck etkisi olan termoelektrik jeneratörler, otomobiller, fabrikalar ve benzer kaynaklardan atılan atık ısının geri kazanılması ile güç üretiminde kullanılmaktadır. 

Bu çalışmada, her biri 8 çift dikdörtgenler prizması ($\mathrm{TEG_{RP}}$) ve silindirik ($\mathrm{TEG_C}$) ayaklardan oluşan ve yüksek sıcaklıkta çalışan termoelektrik jeneratörler üretilmiş ve deneysel olarak incelenmiştir. Bu kapsamda, 19 farklı kompozisyon arasından, sırasıyla değerleri 0.57 ve 0.17 olan ve en yüksek termoelektrik verime ($zT$) sahip Ca$_{2.5}$Ag$_{0.3}$Eu$_{0.2}$Co$_4$O$_9$ ve Zn$_{0.96}$Al$_{0.02}$Ga$_{0.02}$O oksit TE malzemeler seçilmiştir. TE tozlar sol-jel metodu ile sentezlenmiş, $p-$ ve $n-$tipi ayakların konsolidasyonu için soğuk presleme yöntemi kullanılmıştır. Termoelektrik jeneratörler üretilmeden önce Tepki Yüzeyi Metodolojisi ile her iki jeneratör için jeneratör ayaklarının geometrik optimizasyonu gerçekleştirilmiş ve ayak boyutları belirlenmiştir.

Termoelektrik jeneratörlerin güç üretim karakterizasyonu için bir ölçüm sistemi kurularak $I-V$ ve $I-P$ eğrileri elde edilmiştir. Sonuçlar incelendiğinde açık devre voltajı ($V_{OC}$) ve maksimum çıkış gücü ($P_{max}$), sıcaklık farkı ($\Delta T$) arttıkça her iki jeneratör için de artmıştır. $\mathrm{TEG_{RP}}$ ve $\mathrm{TEG_C}$ için maksimum $V_{OC}$, 440~$^\circ$C sıcaklık farkı ve 495~$^\circ$C sıcak yüzey sıcaklığı için sırasıyla 133.1~mV ve 158.2~mV olarak elde edilmiştir. Akım arttıkça çıkış gücü de $P \propto I^2$ orantısına bağlı olarak üretilmiştir. Aynı sıcaklık şartlarında $\mathrm{TEG_{RP}}$ ve $\mathrm{TEG_C}$ için $P_{max}$, sırasıyla 33.7~mW ve 45.5~mW olarak elde edilmiştir. 

$\mathrm{TEG_{RP}}$ ve $\mathrm{TEG_C}$’nin güç üretim performanslarının ve sıcaklık dağılımlarının belirlenebilmesi için kararlı hal ısıl ve ısıl-elektrik analizler gerçekleştirilmiştir. Deneysel sonuçlarda olduğu gibi $\Delta T$ arttıkça $V_{OC}$ ve $P_{max}$ her iki jeneratör için de artmıştır. Her iki jeneratör için de maksimum $V_{OC}$, 410~$^\circ$C sıcaklık farkı ve 495~$^\circ$C sıcak yüzey sıcaklığı için 1117~mV olarak belirlenmiştir. $I-P$ eğrileri her iki jeneratör için de deneysel sonuçlar ile benzer karakteristik göstermiştir. Sonuç olarak, aynı sıcaklık şartlarında $\mathrm{TEG_{RP}}$ ve $\mathrm{TEG_C}$ için $P_{max}$ sırasıyla 3315.1~mW ve 2632.9~mW olarak elde edilmiştir. Isıl-elektrik analizlerin sonuçları üretilen jeneratörlerin deneysel sonuçları ile karşılaştırıldığında ölçülen açık devre voltajı ve çıkış güçlerinin ısıl-elektrik analizleri sonuçlarının çok daha altında olduğu belirlenmiştir. Çalışmadaki bu kayda değer voltaj ve çıkış gücü kayıplarının TE ayaklar ile gümüş iletkenler arasındaki temas dirençlerinden kaynaklandığı düşünülmektedir.
}
% anahtar kelimelerinizin Türkçe çevirilerini virgül ile ayırarak yazınız
\thesisKeywordsTR{Termoelektrik, oksit malzemeler, malzeme karakterizasyonu, termoelektrik özellikler, termoelektrik jeneratör, tepki yüzeyi optimizasyonu, termal-elektrik analiz, güç üretimi.}

\thesisScienceCode{914.1.233} % write your science code here, it will automatically be transferred to the  abstract part as well

%write your acknowledment within curly braces
\thesisAcknowledgment{
I would like to present my foremost thanks to \dots

I appreciate to \dots

My intense gratefulness goes to \dots
}

% write your CV within curly braces
\thesisResume{
write your CV here
}

\thesisSymbolList{ %% add symbols below, you add or remove as needed
\thesisSymbol{$zT$}{Figure of merit}
\thesisSymbol{$PF$}{Power factor}
\thesisSymbol{$S$}{Seebeck coefficient}
\thesisSymbol{$\rho$}{Electrical resistivity}
\thesisSymbol{$\rho_n$}{Electrical resistivity of n-type legs}
\thesisSymbol{$\rho_p$}{Electrical resistivity of p-type legs}
\thesisSymbol{$\sigma$}{Electrical conductivity}
\thesisSymbol{$\Delta T$}{Temperature difference}
\thesisSymbol{$\kappa$}{Thermal conductivity}
\thesisSymbol{$\kappa_E$}{Electron thermal conductivity}
\thesisSymbol{$\kappa_L$}{Lattice thermal conductivity}
\thesisSymbol{$\kappa_{PH}$}{Phonon thermal conductivity}
\thesisSymbol{$T_H$}{Hot side temperature of thermoelectric generator}
\thesisSymbol{$T_C$}{Cold side temperature of thermoelectric generator}
\thesisSymbol{$P$}{Generated output power}
\thesisSymbol{$P_{\text{max}}$}{Maximum generated output power}
\thesisSymbol{$V$}{Electrical potential}
\thesisSymbol{$V_{OC}$}{Open circuit voltage}
\thesisSymbol{$I$}{Electric current}
\thesisSymbol{$J$}{Current density}
\thesisSymbol{$E$}{Electric field intensity}
\thesisSymbol{$B$}{Magnetic field}
\thesisSymbol{$R$}{Electrical resistance}
\thesisSymbol{$R_{in}$}{Internal combined electrical resistance}
\thesisSymbol{$l$}{Distance between contacts}
\thesisSymbol{$N$}{Number of thermoelectric legs}
\thesisSymbol{$V_H$}{Hall voltage}
\thesisSymbol{$R_H$}{Hall coefficient}
\thesisSymbol{$k_B$}{Boltzmann constant}
\thesisSymbol{$h$}{Planck’s constant}
\thesisSymbol{$e$}{Electron charge}
\thesisSymbol{$m^*$}{Effective mass of charge carrier}
\thesisSymbol{$n$}{Carrier concentration}
\thesisSymbol{$E_F$}{Fermi energy}
\thesisSymbol{$L$}{Lorenz number}
\thesisSymbol{$\mu$}{Electron mobility}
\thesisSymbol{$\pi$}{Peltier coefficient}
\thesisSymbol{$\beta$}{Thomson coefficient}
\thesisSymbol{$q$}{Heat flux}
\thesisSymbol{$Q$}{Heat flow}
\thesisSymbol{$\alpha$}{Thermal diffusivity}
\thesisSymbol{$\rho$}{Density}
\thesisSymbol{$c_p$}{Specific heat capacity}
\thesisSymbol{$c_v$}{Specific heat capacity of solids}
\thesisSymbol{$R$}{Gas constant}
\thesisSymbol{$\dfrac{dt}{dx}$}{Temperature gradient}
\thesisSymbol{$N$}{Number of atoms}
\thesisSymbol{$M$}{Molar mass}
\thesisSymbol{$A$}{Cross-sectional area}
\thesisSymbol{$L$}{Length or leg height}
\thesisSymbol{$d$}{Sample thickness}
\thesisSymbol{$y$}{Response variable yield}
\thesisSymbol{$\zeta_1$}{Reaction time}
\thesisSymbol{$\zeta_2$}{Reaction temperature}
\thesisSymbol{$\lambda$}{Wavelength}
\thesisSymbol{$D$}{Average crystalline size}
\thesisSymbol{$\beta$}{Full width at half maximum intensity}
\thesisSymbol{$\theta$}{Bragg’s diffraction angle}
\thesisSymbol{$\alpha$}{Thermal expansion coefficient}
\thesisSymbol{$C$}{Specific heat capacity}
\thesisSymbol{$\dot{q}$}{Heat generation rate}
\thesisSymbol{$D$}{Electric flux density}
\thesisSymbol{$\Pi$}{Peltier coefficient}
\thesisSymbol{$\nabla T$}{Temperature gradient}
\thesisSymbol{$T_s$}{Surface temperature}
\thesisSymbol{$h$}{Convection heat transfer coefficient}
\thesisSymbol{$T_{\infty}$}{Temperature of the medium}
}
\thesisAbbrList{ %% add abbreviations below, you can add or remove as needed
\thesisAbbr{TE}{Thermoelectric}
\thesisAbbr{TEG}{Thermoelectric Generator}
\thesisAbbr{FEM}{Finite Element Method}
\thesisAbbr{FVM}{Finite Volume Method}
\thesisAbbr{HIP}{Hot Isostatic Pressing}
\thesisAbbr{CIP}{Cold Isostatic Pressing}
\thesisAbbr{HP}{Hot Pressing}
\thesisAbbr{CP}{Cold Pressing}
\thesisAbbr{SPS}{Spark Plasma Sintering}
\thesisAbbr{DTA-TG}{Differential Thermal Analysis–Thermogravimetry}
\thesisAbbr{XRD}{X-ray Diffraction}
\thesisAbbr{XPS}{X-ray Photoelectron Spectroscopy}
\thesisAbbr{SEM}{Scanning Electron Microscopy}
\thesisAbbr{TEG$_\mathrm{RP}$}{Thermoelectric generator with rectangular prism legs}
\thesisAbbr{TEG$_\mathrm C$}{Thermoelectric generator with cylindrical legs}
\thesisAbbr{DSC}{Differential Scanning Calorimetry}
\thesisAbbr{DOE}{Design of Experiments}
\thesisAbbr{RSM}{Response Surface Methodology}
\thesisAbbr{BE}{Binding Energy}
\thesisAbbr{FWHM}{Full Width at Half Maximum Intensity}
}
\makeThesisPreamble  %   DO NOT DELETE or change this, you can close it temporarily
\pagenumbering{arabic} % DO NOT DELEYE or change this