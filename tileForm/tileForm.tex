\chapter{Tiling Hubbard}
\section{The Philosophy of Auxiliary Models}

In the previous chapter, we extended the Anderson impurity model to host a localisation-delocalisation transition on the impurity site that captures the phenomenology of the Hubbard dimension in infinite dimensions (as seen from, say, dynamical mean-field theory (DMFT)). In this chapter, we extend this approach by developing a formalism to study the more realistic case of finite-dimensional interacting models by using appropriately chosen impurity models as auxiliary models. The key to an {\it auxiliary model method} is to construct an appropriate auxiliary model that while being tractable captures some essential property of the full lattice model~\cite{martin_2016}. This involves separating the complete degrees of freedom into a system part (\(H_S\)) which one treats exactly and the rest of the system (\(H_R\)) that one might simplify in order to be able to solve the problem. The non-trivial nature of the problem arises of course from the hybridisation \(H_{SR}\) between the system and the bath:
\begin{equation}\begin{aligned}
	{H} &= \begin{bmatrix} & {H}_{R} && {H}_{RS} & \\ & {H}_{RS}^* && {H}_{S} & \end{bmatrix}~.
\end{aligned}\end{equation}

This separation can always be done exactly, if one does not care about tractability. For example, one can take the Hubbard model that describes electrons hopping on a lattice and interacting when two electrons reside on the same lattice site:
\begin{equation}\begin{aligned}
	H_\text{hub} = -t\sum_{\langle i,j \rangle,\sigma }(c^\dagger_{i,\sigma}c_{j,\sigma} + \text{h.c.}) + U\sum_{i}n_{i \uparrow}n_{i \downarrow}~,
\end{aligned}\end{equation}
and separate this into parts, the system being the local orbitals of a specific site \(l\) and the rest of the system being all the other sites:
\begin{equation}\begin{aligned}
	H_S &= U n_{l \uparrow} n_{l \downarrow},\\
	H_R &= -t\sum_{\langle i,j \rangle,\sigma }^\prime(c^\dagger_{i,\sigma}c_{j,\sigma} + \text{h.c.}) + U\sum_{i}^\prime n_{i \uparrow}n_{i \downarrow},\\
	H_{SR} + H_{RS} &= -t\sum_{i,\sigma} (c^\dagger_{l,\sigma}c_{i,\sigma} + \text{h.c.}~,
\end{aligned}\end{equation}
where the primed summation is carried out over all nearest-neighbour pairs that do not involve \(l\) and the summation in \(H_{SR} + H_{RS}\) is over all sites adjacent to \(l\).

The smaller system \(H_S\) (often called the {\it cluster}) is typically chosen such that its eigenstates are known exactly. Progress is then made by choosing a simpler form for \(H_R\) and its coupling \(H_{RS}\) with the smaller system. This combination of the cluster and the simpler bath is then called the \textit{auxiliary system}.
A tractable auxiliary system for the Hubbard model is the single-impurity Anderson model (SIAM); the correlated impurity site represents the set of local orbitals of any given site, while the conduction bath represents the rest of the lattice sites in an approximate fashion.  Such a construction is shown in fig.~\ref{cluster-bath}.
\begin{figure}[!htb]
	\centering
    \includegraphics[width=0.48\textwidth]{clusterBath.pdf}
	\caption{\textit{Left}: Full Hubbard model lattice with onsite repulsion $U^H$ on all sites and hopping between nearest neighbour sites with strength $t^H$. \textit{Right}: Extraction of the auxiliary (cluster+bath) system from the full lattice. The central site on left becomes the impurity site (red) on the right (with an onsite repulsion $\epsilon_d$), while the rest of the $N-1$ sites on the left form a conduction bath (green circles) (with dispersion $\epsilon_k$ and correlation modelled by the self-energy $\Sigma_k(\omega)$) that hybridizes with the impurity through the coupling $V$.}
	\label{cluster-bath}
\end{figure}

It should be noted that any reasonable choice of the cluster and bath would break the translational symmetry of the full model: To allow computing quantities, one would need to make the bath (which is a much larger system) simpler than the cluster (which is a single site). This distinction breaks the translational symmetry of the Hubbard model. As a result, calculations that make use of the translation symmetry of the full model (such as \(k-\)space objects) will require a periodisation operation of an auxiliary model quantity.

The present work is aimed towards developing a new auxiliary model method to study correlated electronic systems. As described above, the first step is to identify the correct impurity model which can faithfully capture the physics of the lattice model we are trying to solve. The local behaviour of this impurity model should reflect the essential local physics of the lattice model. Typically, we will consider impurity model geometries where the real-space bath site connected directly to the impurity hosts some local interaction. We will henceforth refer to this site as the zeroth site. This model must be solved using an impurity solver. In the present work, we use the recently-developed unitary renormalisation group method~\cite{anirbanmott1,anirbanmott2,anirbanurg1,anirbanurg2,siddharthacpi,santanukagome}. The leap to the bulk model is then made by applying manybody lattice translation operators on the auxiliary model. This process, referred to as {\it tiling} here, allows us to reconstruct the lattice model Hamiltonian from that of the auxiliary model. Once the Hamiltonians can be mapped onto one another, it is possible to relate the eigenstates and correlations functions between the auxiliary model and lattice model, using a manybody version of Bloch's theorem.

\subsection{Extending the Anderson impurity model: Identifying the correct auxiliary model}\label{identifyModel}
The standard Anderson model consists of a correlated impurity site coupled with a non-interacting conduction bath. The double occupancy cost on the impurity is \(U\), while the single-particle hopping strength between the impurity and the conduction bath is \(V\). If the impurity site hybridises with a {\it non-interacting} bath defined by a uniform density of states, the impurity spectral function is found to have a well-defined Kondo resonance at low temperatures. Such a model does not exhibit a phase transition; the low-energy phase is one of strong-coupling for all parameter regimes. Increasing the impurity correlation \(U\) only serves to reduce the width of the central peak at the cost of the appearance of side bands at energy scales of the order of \(U\), but the resonance never dies. The situation is however different if the impurity is embedded in a correlated conduction bath with a non-trivial density of states. For the case of a conduction band with the DOS shown in the right of the figure below, the impurity hybridises into a reduced bandwidth because of the correlation on the lattice~\cite{held_2013}.

This difference in the type of conduction baths is utilised in dynamical mean-field theory to describe various phases of the bulk system.
This is done through the DMFT algorithm: one starts with a non-interacting bath, but depending on the value of \(U\), the conduction bath then gets modified and we ultimately end up with something that is different from what we started with.
For small \(U\), the bath does not change much and we retain the central resonance of the impurity spectral function.
This then describes a metal in the bulk.
For larger values of \(U\), however, the bath changes significantly such that its density of states becomes non-constant.
Above a critical \(U_c\), the impurity spectral function gets gapped out, and that then describes the insulating phase in the bulk.
\textit{This leaves open the following question: What is the minimal correlation one can insert into the non-interacting bath (of a single-impurity Anderson model) that can capture both the metallic and insulating phases of the bulk model?}

We have recently studied an extended Anderson impurity model (e-SIAM) where we introduced an explicit Kondo coupling \(J\) and a local correlation \(W\) on the bath zeroth site, which is the site connected to the impurity (see \cite{gazizovaleblanc2023} for some recent findings of non-local effective attractive interactions within the Hubbard model). The Hamiltonian of the e-SIAM for a half-filled impurity site is of the form
\begin{equation}\begin{aligned}
	\label{siam_attr}
	{H}_\text{E-S} = {H}_\text{cbath} + {H}_\text{imp} + {H}_\text{imp-cbath}~,
\end{aligned}\end{equation}
where 
\begin{itemize}
	\item \({H}_\text{cbath} = -\frac{1}{\mathcal{Z}}t\sum_{i=0,1,\ldots;\sigma}\left(c^\dagger_{i,\sigma}c_{i+1,\sigma} + \text{h.c.}\right) - \frac{1}{2}W\left(\hat n_{0 \uparrow} - \hat n_{0 \downarrow}\right)^2\) is the Hamiltonian of the conduction bath consisting of a kinetic energy term and some local interaction terms on the zeroth site,
	\item \({H}_\text{imp} = - \frac{U}{2}\left(\hat n_{d \uparrow} - \hat n_{d \downarrow} \right) ^2\) is the Hamiltonian for the localised impurity site, and
	\item \({H}_\text{imp-bath} = J {\bf S}_d\cdot{\bf S}_0 - V \sum_{\sigma} \left(c^\dagger_{0\sigma} c_{d\sigma} + h.c.\right)\) describes the interaction between the impurity orbitals and the conduction bath.
\end{itemize}
Here, \(c_{d\sigma}\) is the impurity electron operator, \(c_{i\sigma}\) is the conduction bath electron operator, \(c_{0,\sigma}\) is the bath zeroth site operator, \({\bf S}_d\) is the impurity spin operator and \({\bf S}_0 = \sum_{\alpha,\beta}{\bf \sigma}c^\dagger_{0,\alpha}c_{0,\beta}\) is the operator for the local spin in the conduction bath. We have found that the e-SIAM has a stable local moment phase for \(W < -J/4\) with an antiferromagnetic Kondo coupling (\(J > 0\)). We have also shown that this model captures much of the phenomenology of the infinite dimensional Hubbard model (as discovered via DMFT), such as a second-order phase transition at \(T=0\) and the presence of an optical gap in the local spectral function beyond a certain value of interaction strength. Note that the impurity site and the conduction bath are both at half-filling.

\subsection{Tiling towards a Hubbard-Heisenberg model with an embedded extended SIAM}
In this subsection, we provide an explicit example of constructing a lattice model. We will consider a slightly more generalised version of the extended SIAM described in the previous section, where the impurity site coupled to the conduction bath purely through the s-wave channel. We will show that that model leads to a form of a Hubbard-Heisenberg model upon restoring translation invariance via repeated translation operations. The generalisation involve allowing an arbitrary filling on the impurity site and in the conduction bath, through two additional parameter: (i) a particle-hole asymmetry parameter \(\eta\) for the impurity site, (ii) a chemical potential for the conduction bath, and (iii) embedding the impurity into the lattice of the 2D conduction bath. This modified impurity model is shown in Fig.~\ref{embeddedEsiam}. The first term modifies the impurity Hamiltonian \({H}_\text{imp}\) into
\begin{equation}\begin{aligned}\label{onsiteHamiltonian}
	{H}_\text{imp} = -\frac{U}{2}\left(\hat n_{{\bf r}_d \uparrow} - \hat n_{{\bf r}_d \downarrow} \right)^2 - \eta\sum_\sigma \hat n_{{\bf r}_d\sigma}~.
\end{aligned}\end{equation}
where we have placed the impurity site at the position \({\bf r}_d\). The second term modifies the conduction bath Hamiltonian \({H}_\text{cbath}\):
\begin{equation}\begin{aligned}\label{bathHamiltonian}
	{H}_\text{cbath} = -\frac{1}{\sqrt\mathcal{Z}}t\sum_{\left<{\bf r}_i, {\bf r}_j\right>\neq{\bf r}_d;\sigma}\left(c^\dagger_{{\bf r}_i,\sigma}c_{{\bf r}_j,\sigma} + \text{h.c.}\right) - \frac{1}{2 \mathcal{Z}}W\sum_{{\bf z}\in\text{NN}({\bf r}_d)}\left(\hat n_{{\bf z}, \uparrow} - \hat n_{{\bf z}, \downarrow}\right)^2 - \mu \sum_{{\bf r}_i \neq {\bf r}_d}\hat n_{{\bf r}_i,\sigma}~,
\end{aligned}\end{equation}
where \(\left<{\bf r}_i, {\bf r}_j\right>\neq {\bf r}_d\) indicates that the sum is over all nearest-neighbour pairs of sites avoiding the impurity site \({\bf r}_d\).

In this notation, the interaction Hamiltonian can be written as
\begin{equation}\begin{aligned}\label{interactionHamiltonian}
	{H}_\text{imp-cbath} = \frac{J}{\mathcal{Z}}\sum_{\sigma,\sigma^\prime}\sum_{{\bf z}\in\text{NN}({\bf r}_d)} {\bf S}_{{\bf r}_d}\cdot{\boldsymbol \tau}_{\sigma,\sigma^\prime}c^\dagger_{{\bf z},\sigma}c_{{\bf z},\sigma^\prime} - \frac{V}{\sqrt{\mathcal{Z}}}\sum_{\sigma} \sum_{{\bf z}\in\text{NN}({\bf r}_d)}\left(c^\dagger_{{\bf r}_d,\sigma} c_{{{\bf z}},\sigma} + h.c.\right)
\end{aligned}\end{equation}
where \(\boldsymbol \tau = \left( \tau_x, \tau_y, \tau_z \right) \) is the vector of Pauli matrices. \(\sigma\) and \(\sigma^\prime\) can be \(\pm 1\) and represent up and down configurations.

We now follow the prescription laid out in eq.~\ref{tilingPrescription}. The tiled Hamiltonian can be written as 
\begin{equation}\begin{aligned}
	{H}_\text{tiled} = \sum_{{\bf r}}T^\dagger({\bf r} - {\bf r}_d)\left[{H}_\text{cbath} + {H}_\text{imp} + {H}_\text{imp-cbath}\right]T({\bf r} - {\bf r}_d)~.
\end{aligned}\end{equation}
Note that in comparison to eq.~\ref{tilingPrescription}, we have dropped the sum over the zeroth sites, because our impurity model Hamiltonian (defined using eqs.~\ref{onsiteHamiltonian} through \ref{interactionHamiltonian}) already contains a sum over these zeroth sites.

We consider the effect of the translation operations on each part of the Hamiltonian. We first have
\begin{equation}\begin{aligned}
	&\sum_{{\bf r}}T^\dagger({\bf r} - {\bf r}_d){H}_\text{cbath}({\bf r}_d, {\bf z})T({\bf r} - {\bf r}_d) \\
	=& \sum_{{\bf r}}T^\dagger({\bf r} - {\bf r}_d)\left[-\frac{1}{\sqrt\mathcal{Z}}t\sum_{\left<{\bf r}_i, {\bf r}_j\right>\neq{\bf r}_d;\sigma}\left(c^\dagger_{{\bf r}_i,\sigma}c_{{\bf r}_j,\sigma} + \text{h.c.}\right) - \frac{W}{2\mathcal{Z}}\sum_{{\bf z}\in\text{NN}({\bf r}_d)}\left(\hat n_{{\bf z}, \uparrow} - \hat n_{{\bf z}, \downarrow}\right)^2 - \mu \sum_{{\bf r}_i \neq {\bf r}_d}\hat n_{{\bf r}_i,\sigma}\right]T({\bf r} - {\bf r}_d) \\
	=& -\frac{1}{\sqrt\mathcal{Z}}t\sum_{\left<{\bf r}_i, {\bf r}_j\right>;\sigma}\sum_{{\bf r}\neq{\bf r}_i,{\bf r}_j}\left(c^\dagger_{{\bf r}_i,\sigma}c_{{\bf r}_j,\sigma} + \text{h.c.}\right) - \frac{W}{2\mathcal{Z}}\sum_{\bf r}\sum_{{\bf z}\in\text{NN}\left({\bf r}\right)}\left(\hat n_{{\bf z}, \uparrow} - \hat n_{{\bf z}, \downarrow}\right)^2 - \mu \sum_{{\bf r}_i}\sum_{{\bf r} \neq {\bf r}_i}\hat n_{{\bf r}_i,\sigma}\\
	=& -\frac{1}{\sqrt\mathcal{Z}}(N-2)t\sum_{\left<{\bf r}_i, {\bf r}_j\right>;\sigma}\left(c^\dagger_{{\bf r}_i,\sigma}c_{{\bf r}_j,\sigma} + \text{h.c.}\right) - \frac{1}{2}W\sum_{\bf r}\left(\hat n_{{\bf r} , \uparrow} - \hat n_{{\bf r} , \downarrow}\right)^2 - \mu (N-1) \sum_{{\bf r}}\hat n_{{\bf r},\sigma}\\
\end{aligned}\end{equation}
In the first step, the factor of \(\mathcal{Z}\) is cancelled out by the trivial sum over \({\bf r}_0\) in the first and third terms. At the last step, the three terms simplified for the following reasons. The inequality \(\left<{\bf r}_i, {\bf r}_j\right>\neq{\bf r}\) in the first term ensures that each nearest-neighbour pair appears in \(N-2\) instances of the auxiliary model, \(N\) being the total number of lattice sites; the two instances that do not contribute are the ones in which the impurity site itself is at \({\bf r}_i\) or \({\bf r}_j\). For the second term, the double sum \(\sum_{\bf r}\sum_{{\bf z} \in \text{NN}({\bf r})}\) evaluates to \(\mathcal{Z}\sum_{\bf r}\), because each point on the lattice appears \(\mathcal{W}\) times in the summation. This factor of \(\mathcal{W}\) cancels the one in the denominator. In the third term, the inner summation simply evaluates to \(N-1\), and we finally replace the dummy index \({\bf r}_i\) with \({\bf r}\). 

The next part is
\begin{equation}\begin{aligned}
	\sum_{{\bf r}}T^\dagger({\bf r} - {\bf r}_d){H}_\text{imp}T({\bf r} - {\bf r}_d) = -\frac{U}{2}\sum_{{\bf r}}\left(\hat n_{{\bf r} \uparrow} - \hat n_{{\bf r} \downarrow} \right)^2 - \eta\sum_{{\bf r},\sigma} \hat n_{{\bf r}\sigma}~.
\end{aligned}\end{equation}
This is obtained simply by replacing the impurity position \({\bf r}_d\) with the translated position \({\bf r}\), generating a translation-invariant Hubbard term (the first term) and a finite chemical potential (second term).

We now consider the final term:
\begin{equation}\begin{aligned}
	&\sum_{{\bf r}}T^\dagger({\bf r} - {\bf r}_d){H}_\text{imp-cbath}T({\bf r} - {\bf r}_d)\\
	&=\sum_{{\bf r}}T^\dagger({\bf r} - {\bf r}_d)\left[\frac{1}{\mathcal{Z}}\sum_{{\bf z}\in\text{NN}({\bf r}_d)}\sum_{\sigma,\sigma^\prime}J {\bf S}_{{\bf r}_d}\cdot{\boldsymbol \tau}_{\sigma,\sigma^\prime}c^\dagger_{{\bf z},\sigma}c_{{\bf z},\sigma^\prime} - \frac{1}{\sqrt\mathcal{Z}}\sum_{{\bf z}\in\text{NN}({\bf r}_d)}\sum_{\sigma} V \left(c^\dagger_{{\bf r}_d,\sigma} c_{{{\bf z}},\sigma} + h.c.\right)\right]T({\bf r} - {\bf r}_d)\\
	&=\sum_{{\bf r}}\sum_{{\bf z}\in\text{NN}({\bf r})}\left[\frac{1}{\mathcal{Z}}\sum_{\sigma,\sigma^\prime}J {\bf S}_{{\bf r}}\cdot{\boldsymbol \tau}_{\sigma,\sigma^\prime}c^\dagger_{{\bf z},\sigma}c_{{\bf z},\sigma^\prime} \right. -  \frac{1}{\sqrt\mathcal{Z}}\left.\sum_{\sigma} V\left(c^\dagger_{{\bf r}_d,\sigma} c_{{{\bf z}},\sigma} + h.c.\right)\right]\\
					 &=\sum_{\left< {\bf r}_i, {\bf r}_j\right>}\left[\frac{2}{\mathcal{Z}}J {\bf S}_{{\bf r}_i}\cdot{\bf S}_{{\bf r}_j} - \frac{2}{\sqrt\mathcal{Z}}V \sum_{\sigma}\left(c^\dagger_{{\bf r}_i,\sigma} c_{{{\bf r}_j},\sigma} + h.c.\right)\right]\\
\end{aligned}\end{equation}
At the last step, each nearest-neighbour pair of sites \({\bf r}_i, {\bf r}_j\) appear 2 times in the summation, because any site is a member of two distinct nearest-neighbour pairs. We have also defined \({\bf S}_{{\bf r}_j} =\sum_{\sigma,\sigma^\prime} {\boldsymbol \tau}_{\sigma,\sigma^\prime}c^\dagger_{{\bf z},\sigma}c_{{\bf z},\sigma^\prime}\) as the local spin operator.

The total tiled Hamiltonian is therefore
\begin{equation}\begin{aligned}
	{H}_\text{tiled} = -\frac{N-2}{\sqrt\mathcal{Z}}t\sum_{\left<{\bf r}_i, {\bf r}_j\right>,\sigma}\left(c^\dagger_{{\bf r}_i,\sigma}c_{{\bf r}_j,\sigma} + \text{h.c.}\right) - \mu (N-1) \sum_{{\bf r}}\hat n_{{\bf r},\sigma} - \frac{1}{2}W\sum_{\bf r}\left(\hat n_{{\bf r} , \uparrow} - \hat n_{{\bf r} , \downarrow}\right)^2  -\frac{U}{2}\sum_{{\bf r}}\left(\hat n_{{\bf r} \uparrow} - \hat n_{{\bf r} \downarrow} \right)^2 - \eta\sum_{{\bf r},\sigma} \hat n_{{\bf r}\sigma}\\
	+ \sum_{\left< {\bf r}_i, {\bf r}_j\right>}\left[\frac{2}{\mathcal{Z}}J {\bf S}_{{\bf r}_i}\cdot{\bf S}_{{\bf r}_j} - \frac{2}{\sqrt\mathcal{Z}}V \sum_{\sigma}\left(c^\dagger_{{\bf r}_i,\sigma} c_{{{\bf r}_j},\sigma} + h.c.\right)\right]
\end{aligned}\end{equation}
While constructing the tiled Hamiltonian, we have added extra copies of the non-interacting Hamiltonian \({H}_\text{cbath-nint} = -\frac{1}{\sqrt\mathcal{Z}}t\sum_{\left<{\bf r}_i, {\bf r}_j\right>;\sigma}\left(c^\dagger_{{\bf r}_i,\sigma}c_{{\bf r}_j,\sigma} + \text{h.c.}\right) - \mu \sum_{{\bf r}}\hat n_{{\bf r},\sigma}\) for the conduction bath (this results in the factors of \(N-2\) and \(N-1\) in front of the first and third terms). Upon removing these repeated terms, the tiled Hamiltonian becomes
\begin{equation}\begin{aligned}
	{H}_\text{tiled} &= \sum_{{\bf r}}{H}_\text{aux}({\bf r}) - (N-3){H}_\text{cbath-nint}\\
							 &= -\frac{1}{\sqrt\mathcal{Z}}(t + 2V)\sum_{\left<{\bf r}_i, {\bf r}_j\right>;\sigma}\left(c^\dagger_{{\bf r}_i,\sigma}c_{{\bf r}_j,\sigma} + \text{h.c.}\right) - \frac{1}{2}\left(U + W\right)\sum_{\bf r}\left(\hat n_{{\bf r} , \uparrow} - \hat n_{{\bf r} , \downarrow}\right)^2 - \left(\eta + 2\mu\right) \sum_{{\bf r}}\hat n_{{\bf r},\sigma} + \frac{2}{\mathcal{Z}}J\sum_{\left< {\bf r}_i, {\bf r}_j\right>} {\bf S}_{{\bf r}_i}\cdot{\bf S}_{{\bf r}_j}
\end{aligned}\end{equation}

The result of the tiling operations is a Hubbard-Heisenberg model, of the form
\begin{equation}\begin{aligned}
	{H}_\text{HH} &= -\frac{1}{\sqrt\mathcal{Z}}\tilde t\sum_{\left<{\bf r}_i, {\bf r}_j\right>;\sigma}\left(c^\dagger_{{\bf r}_i,\sigma}c_{{\bf r}_j,\sigma} + \text{h.c.}\right) - \tilde \mu \sum_{{\bf r}}\hat n_{{\bf r},\sigma} + \frac{1}{\mathcal{Z}}\tilde J\sum_{\left< {\bf r}_i, {\bf r}_j\right>}{\bf S}_{{\bf r}_i}\cdot{\bf S}_{{\bf r}_j} - \frac{1}{2}\tilde U\sum_{\bf r}\left(\hat n_{{\bf r} , \uparrow} - \hat n_{{\bf r} , \downarrow}\right)^2  ~,
\end{aligned}\end{equation}
where the tilde symbol indicates that the parameters are for the lattice model (and not the auxiliary model). By comparing the tiled model and the general lattice model, the lattice model parameters and the auxiliary model parameters can be mapped to each other:
\begin{equation}\begin{aligned}\label{couplingsMappings}
	\tilde t = t+2V,~~ \tilde U = U + W, ~ ~ \tilde \mu = 2\mu + \eta,~ ~ \tilde J = 2J~.
\end{aligned}\end{equation}
In summary, the appropriate method for reconstructing the lattice model Hamiltonian is therefore
\begin{equation}\begin{aligned}\label{tilingPrescriptionFinal}
	{H}_\text{tiled} &= \sum_{{\bf r}}{H}_\text{aux}({\bf r}) - N{H}_\text{cbath-nint}~,
\end{aligned}\end{equation}
where we replaced \(N-3\) with \(N\) assuming a large number of sites. Using this, the extended-SIAM gets ``expanded" into a Hubbard-Heisenberg model.

\subsection{Properties of Translation Operators}

In order to create the bulk model, we now need to translate this auxiliary model over the entire lattice. For this, we define {\it many-particle global} translation operators \(T({\bf a})\) that translate all positions by a vector \({\bf a}\). In terms of manybody states and operators, their action is defined as 
\begin{equation}\begin{aligned}\label{translationDefinition}
	&T^\dagger({\bf a}) \ket{{\bf r}_1, {\bf r}_2,\ldots} = \ket{{\bf r}_1 + {\bf a}}\otimes\ket{{\bf r}_2 + {\bf a}}\ldots\otimes\ket{{\bf r}_n + {\bf a}}\\
	&T^\dagger({\bf a}) \mathcal{O}\left({\bf r}_1, {\bf r}_2,\ldots\right)T({\bf a}) = \mathcal{O}\left({\bf r}_1 + {\bf a}, {\bf r}_2 + {\bf a},\ldots\right)~,
\end{aligned}\end{equation}
where \(\ket{{\bf r}_1, {\bf r}_2,\ldots}\) is a state in the manyparticle Fock-space basis with the particles localised at the specified positions. For example, for a local fermionic creation operator \(c^\dagger({\bf r})\), we have 
\begin{equation}\begin{aligned}
	T^\dagger({\bf a}) c^\dagger({\bf r}) T({\bf a}) = c^\dagger({\bf r} + {\bf a})~.
\end{aligned}\end{equation}
It acts similarly on the auxiliary model Hamiltonian:
\begin{equation}\begin{aligned}
	T^\dagger({\bf a}){H}_\text{aux}({\bf r}_d)T({\bf a}) = {H}_\text{aux}({\bf r}_d + {\bf a})~,
\end{aligned}\end{equation}
translating all sites by the vector \({\bf a}\). By introducing the Fourier transform to momentum space,
\begin{equation}\begin{aligned}
	\ket{{\bf r}_1, {\bf r}_2,\ldots} = \otimes_{j=1}^N \int d{\bf k}_j e^{-i {\bf r}_j\cdot{\bf k}_j} \ket{{\bf k}_j}~,
\end{aligned}\end{equation}
it is easy to see that the total momentum states are eigenstates of the global translation operators:
\begin{equation}\begin{aligned}
	T^\dagger({\bf a})\ket{{\bf k}_1, {\bf k}_2,\ldots} &= \otimes_{j=1}^N \int d{\bf r}_j e^{i {\bf r}_j\cdot{\bf k}_j} T^\dagger({\bf a})\ket{{\bf r}_j}~,\\
							    &=\otimes_{j=1}^N \int d{\bf r}_j e^{i {\bf r}_j\cdot{\bf k}_j} \ket{{\bf r}_j + {\bf a}}\\
							    &=e^{-i {\bf a}_j\cdot{\bf k}_\text{tot}}\ket{{\bf k}_1, {\bf k}_2,\ldots},
\end{aligned}\end{equation}
where \({\bf k}_\text{tot} = \sum_j {\bf k}_j\) is the total momentum.
From the form in eq.~\ref{tilingPrescriptionFinal}, the tiled Hamiltonian is symmetric under global many-body translations of the kind defined in eq.~\ref{translationDefinition}, by arbitrary lattice spacings:
\begin{equation}\begin{aligned}\label{translationSymmetry}
	&T({\bf a})^\dagger\sum_{{\bf r}}{H}_\text{aux}({\bf r})T({\bf a}) = \sum_{{\bf r}}{H}_\text{aux}({\bf r + a}) = \sum_{{\bf r}^\prime}{H}_\text{aux}({\bf r}^\prime)\\
	&T({\bf a})^\dagger\sum_{{\bf r}}{H}_\text{cbath-nint}T({\bf a}) = {H}_\text{cbath-nint}\\
	&\implies T({\bf a})^\dagger{H}_\text{tiled}T({\bf a}) = {H}_\text{tiled}~.
\end{aligned}\end{equation}
In the first equation, we used the fact that the translation operator simply translates the auxiliary model at the position \({\bf r}\) into another one at the position \({\bf r} + {\bf a}\). Since both are part of the summation, the summation remains unchanged. The second equation uses the fact that the Hamiltonian \({H}_\text{cbath-nint}\) is that of a tight-binding model and is therefore translation-invariant. The fact that the Hamiltonian \({H}_\text{tiled}\) commutes with the many-body translation operator implies that the total crystal momentum \(\vec k\) is a conserved quantity.

\subsection{Form of the eigenstates: Bloch's theorem}
In the tight-binding approach to lattice problems, the full Hamiltonian is described by adding the localised Hamiltonians at each site, and the full eigenstate \(\ket{\Psi}\) is then obtained by constructing liner combinations of the eigenstates \(\ket{\psi_i}\) of the local Hamiltonians such that \(\ket{\Psi}\) satisfies Bloch's theorem: \(\ket{\Psi_{\bf k}} = \sum_{i} e^{i {\bf k}\cdot{\bf r}_i} \ket{\psi_i}\), where \({\bf r}_i\) sums over the positions of the local Hamiltonians. Bloch's theorem ensures that eigenstates satisfy the following relation under a translation operation by an arbitrary number of lattice spacings \({n\bf a}\):
\begin{equation}\begin{aligned}
	T^\dagger(n{\bf a})\ket{\Psi_{{\bf k}}} = \sum_{i} e^{i {\bf k}\cdot{\bf r}_i} \ket{\psi_{i + n}} = e^{-in{\bf k}\cdot{\bf a}}\ket{\Psi_{\bf k}}
\end{aligned}\end{equation}
The definition and some properties of these global translation operations were provided in Appendix~\ref{BlochProperties}. It was shown there that they share eigenstates with the total momentum operator. In a lattice model, this continuous symmetry gets lowered to its discrete form: the total {\it crystal} momentum is conserved by any scattering process. As a result, the eigenstates can be labelled using the combined index \(s = \left({\bf k}, n\right) \) where \({\bf k}\) is the total crystal momentum and \(n\) is a band index \(n\).

The eigenstates \(\ket{\Psi_{s}}\) (\(s = \left({\bf k}, n\right)\)) of the lattice Hamiltonians obtained using eq.~\ref{tilingPrescriptionFinal} also enjoy a {\it many-body} Bloch's theorem~\cite{stoyanova}, because the tiling procedure restores the translation symmetry of the Hamiltonian (as shown in eq.~\ref{translationSymmetry}). This means that the {\it local} eigenstates \(\ket{\psi_n\left({\bf r}_d\right)}\) (with the impurity located at an arbitrary position \({\bf r}_d\)) of the unit cell auxiliary model Hamiltonian \({H}_\text{aux}({\bf r}_d)\) defined in eq.~\ref{unitCellHamiltonian} can be used to construct eigenstates of the lattice Hamiltonian. The index \(n(=0,1,\ldots)\) in the subscript indicates that it is the \(n^\text{th}\) eigenstate of the auxiliary model.

The state \(\ket{\psi_n\left({\bf r}_d\right)}\) does not specify the position of the zeroth site, because the unit cell Hamiltonian \({H}_\text{aux}({\bf r}_d)\) itself has been averaged over \(\mathcal{Z}\) zeroth sites. Accordingly, we can express the averaged eigenstate \(\ket{\psi_n\left({\bf r}_d\right)}\) as
\begin{equation}\begin{aligned}
	\ket{\psi_n\left({\bf r}_d\right)} = \frac{1}{\sqrt\mathcal{Z}}\sum_{{\bf z} \in \text{NN}({\bf r}_d)}\ket{\psi_n\left({\bf r}_d, {\bf z}\right)}~,
\end{aligned}\end{equation}
where \(\ket{\psi_n\left({\bf r}_d, {\bf z}\right)}\) is an auxiliary model eigenstate with the impurity and zeroth sites placed at \({\bf r}_d\) and \({\bf z}\). With this in mind, the following unnormalised combination of the auxiliary model eigenstates satisfies a many-particle equivalent of Bloch's theorem~\cite{stoyanova}:
\begin{equation}\begin{aligned}\label{eigenstateProposal}
	\ket{\Psi_{s}} \equiv \ket{\Psi_{{\bf k}, n}} &= \frac{1}{\sqrt N}\sum_{{\bf r}_d} e^{i {\bf k}\cdot{\bf r}_d} \ket{\psi_{n}\left({\bf r}_d\right)} = \frac{1}{\sqrt{\mathcal{Z} N}}\sum_{{\bf r}_d}\sum_{{\bf z} \in \text{NN}({\bf r}_d)} e^{i {\bf k}\cdot{\bf r}_d} \ket{\psi_{n}\left({\bf r}_d, {\bf z}\right)}~,
\end{aligned}\end{equation}
where \(N\) is the total number of lattice sites and \({\bf r}_d\) is summed over all lattice spacings. The set of \(n=0\) states form the lowest band in the spectrum of the lattice, while higher values of \(n\) produce the more energetic bands. The ground state \(s = s_0\) is obtained by setting \({\bf k}\) and \(n\) to 0:
\begin{equation}\begin{aligned}\label{groundstateProposal}
	\ket{\Psi_\text{gs}} \equiv \ket{\Psi_{s_0}} &= \frac{1}{\sqrt N}\sum_{{\bf r}_d} e^{i {\bf k}\cdot{\bf r}_d} \ket{\psi_\text{gs}\left({\bf r}_d\right)} = \frac{1}{\sqrt{\mathcal{Z} N}}\sum_{{\bf r}_d}\sum_{{\bf z} \in \text{NN}({\bf r}_d)} e^{i {\bf k}\cdot{\bf r}_d} \ket{\psi_\text{gs}\left({\bf r}_d, {\bf z}\right)}
\end{aligned}\end{equation}


\section{Mapping Static and Dynamic Correlations from the Auxiliary Model to the Lattice Model}\label{tilingProcedure}
\subsection{One-Particle Greens Functions: Momentum-space}
In the previous part, we proposed a form for the ground state \(\ket{\Psi_\text{gs}}\) of the bulk Hamiltonian in terms of the ground-states \(\ket{\psi_\text{gs}}\) of the auxiliary models.
In this section, we will relate one-particle Greens functions of the bulk lattice to those of the auxiliary model. We will assume that the auxiliary model Hilbert space has the same dimensions as that of the bulk lattice model. We define the retarded time-domain lattice \(k-\)space Greens function at zero temperature as
\begin{equation}\begin{aligned}
	\tilde G({\bf K}\sigma; t) = -i\theta(t) \braket{\Psi_\text{gs} | \left\{ c_{{\bf K}\sigma}(t), c^\dagger_{{\bf K}\sigma} \right\} | \Psi_\text{gs}}~.
\end{aligned}\end{equation}
where the bulk Hamiltonian \(H_\text{tiled}\) leads to the dynamics of the annihilation operators at time \(t\): 
\begin{equation}\begin{aligned}\label{heisenberg}
	c_{{\bf K}\sigma}(t) = e^{it H_\text{tiled} }c_{{\bf K}\sigma}e^{-i t H_\text{tiled}}~.
\end{aligned}\end{equation}
We now proceed to simplify one of the terms of the anticommutator (for simplicity of notation):
\begin{equation}\begin{aligned}\label{greensFunction1}
	&\braket{\Psi_\text{gs} | c_{{\bf K}\sigma}(t) c^\dagger_{{\bf K}\sigma} | \Psi_\text{gs}} = \frac{1}{N^2}\sum_{\vec r,\vec \Delta}e^{-i{\bf K}_0\cdot\vec\Delta}\braket{\psi_{0}(\vec r+\vec \Delta) | c_{{\bf K}\sigma}(t) c^\dagger_{{\bf K}\sigma} | \psi_{0}(\vec r)}~.
\end{aligned}\end{equation}
To make further progress, we insert the identity resolution \(1 = \sum_s\ket{\Psi_s}\bra{\Psi_s}\) in between the two operators, where \(s=\left({\bf k}, n\right)\) sums over all eigenstates (with energies \(\tilde E_s\)). The lattice eigenstates themselves can again be written in terms of those of the auxiliary model, using eq.~\ref{eigenstateProposal}:
\begin{equation}\begin{aligned}
	\ket{\Psi_s}\bra{\Psi_s} = \sum_{\vec r^\prime,\vec \Delta^\prime} e^{i\vec k\cdot\vec\Delta^\prime}\ket{\psi_n(\vec r^\prime+\vec \Delta^\prime)}\bra{\psi_n(\vec r^\prime)}~.
\end{aligned}\end{equation}
With this, eq.~\ref{greensFunction1} becomes
	\begin{equation}\begin{aligned}\label{greensfunction2}
		\braket{\Psi_\text{gs} | c_{{\bf K}\sigma}(t) c^\dagger_{{\bf K}\sigma} | \Psi_\text{gs}} = \frac{1}{N^2}\sum_s\sum_{\vec r,\vec \Delta}\sum_{\vec r^\prime,\vec \Delta^\prime}e^{-i\vec k_0\cdot\vec\Delta} e^{i\vec k\cdot\vec\Delta^\prime}\braket{\psi_{0}(\vec r+\vec \Delta) | c_{{\bf K}\sigma}(t) \ket{\psi_n(\vec r^\prime+\vec \Delta^\prime)}\bra{\psi_n(\vec r^\prime)} c^\dagger_{{\bf K}\sigma} | \psi_{0}(\vec r)}~.
\end{aligned}\end{equation}
In order to bring this expression closer to the form of an auxiliary model Greens function, we would like to transform the initial and final states \(\ket{\psi(\vec r)}\) and \(\ket{\psi(\vec r + \vec \Delta)}\) to apply to the same auxiliary model. This is done by using the relation: \(\ket{\psi(\vec r + \Delta)} = T^\dagger(\vec \Delta)\ket{\psi(\vec r)}\), where \(T^\dagger(\vec \Delta)\) translates all lattice sites by the vector \(\vec\Delta\).
\begin{equation}\begin{aligned}
	\braket{\psi_{0}(\vec r+\vec \Delta) | c_{{\bf K}\sigma}(t) | \psi_n(\vec r^\prime+\vec \Delta^\prime)} = \braket{\psi_{0}(\vec r) | T(\vec \Delta) c_{{\bf K}\sigma}(t) | \psi_n(\vec r^\prime+\vec \Delta^\prime)} = \braket{\psi_{0}(\vec r) | T(\vec \Delta) c_{{\bf K}\sigma}(t) T^\dagger(\vec \Delta) | \psi_n(\vec r^\prime+\vec \Delta^\prime - \vec \Delta)}
\end{aligned}\end{equation}
The effect of the translation operators on the \(k-\)space annihilation operator can be easily ascertained by transforming it to real-space, using the Fourier transform definition
\begin{equation}\begin{aligned}\label{fourTransf}
	c({\bf K}) = \frac{1}{\sqrt N}\sum_{\bf r}e^{-i{\bf K}\cdot{\bf r}}c({\bf r})~.
\end{aligned}\end{equation}
Upon applying this, we get
\begin{equation}\begin{aligned}\label{translationOnField}
	T\left(\vec a\right) c({\bf K}) T^\dagger\left(\vec a\right) = \frac{1}{N}\sum_{\bf r}e^{-i{\bf K}\cdot{\bf r}}T\left(\vec a\right) c({\bf r}) T^\dagger\left(\vec a\right) = \frac{1}{N}\sum_{\bf r}e^{-i{\bf K}\cdot{\bf r}}c({\bf r} - \vec a) = e^{-i{\bf K}\cdot\vec a}c({\bf K})~.
\end{aligned}\end{equation}
Using this identity on the above expression gives
\begin{equation}\begin{aligned}
	\braket{\psi_{0}(\vec r+\vec \Delta) | c_{{\bf K}\sigma}(t) | \psi_n(\vec r^\prime+\vec \Delta^\prime)} = e^{-i{\bf K} \cdot \vec \Delta}\braket{\psi_{0}(\vec r) | c_{{\bf K}\sigma}(t) | \psi_n(\vec r^\prime+\vec \Delta^\prime - \vec \Delta)}
\end{aligned}\end{equation}
Finally replacing this all the way back into eq.~\ref{greensfunction2} gives
	\begin{equation}\begin{aligned}
		\braket{\Psi_\text{gs} | c_{{\bf K}\sigma}(t) c^\dagger_{{\bf K}\sigma} | \Psi_\text{gs}} = \frac{1}{N^2}\sum_s\sum_{\vec r,\vec \Delta}\sum_{\vec r^\prime,\vec \Delta^\prime}e^{-i\left(\vec k_0 + {\bf K}\right)\cdot\vec\Delta} e^{i\vec k\cdot\vec\Delta^\prime}\braket{\psi_{0}(\vec r) | c_{{\bf K}\sigma}(t) \ket{\psi_n(\vec r^\prime+\vec \Delta^\prime - \vec\Delta)}\bra{\psi_n(\vec r^\prime)} c^\dagger_{{\bf K}\sigma} | \psi_{0}(\vec r)}~.
\end{aligned}\end{equation}
To further unify the operators, we make the substitution \(\vec \Delta^\prime \to \vec \Delta^\prime + \vec \Delta\), leading to the expression
	\begin{equation}\begin{aligned}
		\braket{\Psi_\text{gs} | c_{{\bf K}\sigma}(t) c^\dagger_{{\bf K}\sigma} | \Psi_\text{gs}} = \frac{1}{N^2}\sum_s\sum_{\vec r,\vec \Delta}\sum_{\vec r^\prime,\vec \Delta^\prime}e^{-i\left(\vec k_0 + {\bf K} - \vec k\right)\cdot\vec\Delta} e^{i\vec k\cdot\vec\Delta^\prime}\braket{\psi_{0}(\vec r) | c_{{\bf K}\sigma}(t) \ket{\psi_n(\vec r^\prime+\vec \Delta^\prime)}\bra{\psi_n(\vec r^\prime)} c^\dagger_{{\bf K}\sigma} | \psi_{0}(\vec r)}~.
\end{aligned}\end{equation}
The sum over \(\vec\Delta\) can now be carried out, resulting in
\begin{equation}\begin{aligned}\label{greensfunction3}
		\braket{\Psi_\text{gs} | c_{{\bf K}\sigma}(t) c^\dagger_{{\bf K}\sigma} | \Psi_\text{gs}} = \frac{1}{N}\sum_n\sum_{\vec r,\vec r^\prime,\vec \Delta^\prime} e^{i\left( \vec k_0 + {\bf K} \right) \cdot\vec\Delta^\prime}\braket{\psi_{0}(\vec r) | c_{{\bf K}\sigma}(t) \ket{\psi_n(\vec r^\prime+\vec \Delta^\prime)}\bra{\psi_n(\vec r^\prime)} c^\dagger_{{\bf K}\sigma} | \psi_{0}(\vec r)}~,
\end{aligned}\end{equation}
where the sum over \(s=(\vec k, n)\) has been reduced to a sum over the auxiliary model eigenstate index \(n\) because of the Kronecker delta \(\delta\left(\vec k_0 + {\bf K} - \vec k\right)\). This can be further simplified by splitting the sum over \(\vec\Delta^\prime\) into positive and negative parts and then making the transformation \(\vec r^\prime \to \vec r^\prime + \vec \Delta^\prime\):
\begin{equation}\begin{aligned}
	\sum_{\vec r^\prime,\vec \Delta^\prime} e^{i\left( \vec k_0 + {\bf K} \right) \cdot\vec\Delta^\prime}\ket{\psi_n(\vec r^\prime+\vec \Delta^\prime)}\bra{\psi_n(\vec r^\prime)} &= \frac{1}{2}\sum_{\vec r^\prime,\vec \Delta^\prime}\left[ e^{i\left( \vec k_0 + {\bf K} \right) \cdot\vec\Delta^\prime}\ket{\psi_n(\vec r^\prime+\vec \Delta^\prime)}\bra{\psi_n(\vec r^\prime)} + e^{-i\left( \vec k_0 + {\bf K} \right) \cdot\vec\Delta^\prime}\ket{\psi_n(\vec r^\prime-\vec \Delta^\prime)}\bra{\psi_n(\vec r^\prime)}\right] \\
																																												   &= \frac{1}{2}\sum_{\vec r^\prime,\vec \Delta^\prime}\left[ e^{i\left( \vec k_0 + {\bf K} \right) \cdot\vec\Delta^\prime}\ket{\psi_n(\vec r^\prime+\vec \Delta^\prime)}\bra{\psi_n(\vec r^\prime)} + e^{-i\left( \vec k_0 + {\bf K} \right) \cdot\vec\Delta^\prime}\ket{\psi_n(\vec r^\prime)}\bra{\psi_n(\vec r^\prime+\vec \Delta^\prime)}\right] \\
\end{aligned}\end{equation}
For each pair of \(\vec r^\prime\) and \(\vec\Delta^\prime\), the term within the box brackets has the form of a two-level Hamiltonian between the states \(\ket{\psi_n(\vec r^\prime)}\) and \(\ket{\psi_n(\vec r^\prime+\vec \Delta^\prime)}\), with a tunnelling amplitude \(e^{i\left( \vec k_0 + {\bf K} \right) \cdot\vec\Delta^\prime}\). The term can therefore be written in the eigenbasis of this Hamiltonian:
\begin{equation}\begin{aligned}
	\sum_{\vec r^\prime,\vec \Delta^\prime} e^{i\left( \vec k_0 + {\bf K} \right) \cdot\vec\Delta^\prime}\ket{\psi_n(\vec r^\prime+\vec \Delta^\prime)}\bra{\psi_n(\vec r^\prime)} &=\frac{1}{2}\sum_{\vec r^\prime,\vec \Delta^\prime}\left[ \ket{\chi_n^+(\vec r^\prime,\vec\Delta^\prime)}\bra{\chi_n^+(\vec r^\prime,\vec\Delta^\prime)} - \ket{\chi_n^-(\vec r^\prime,\vec\Delta^\prime)}\bra{\chi_n^-(\vec r^\prime,\vec\Delta^\prime)}\right],
\end{aligned}\end{equation}
where \(\ket{\chi_n^\pm(\vec r^\prime,\vec\Delta^\prime)} = \frac{1}{\sqrt 2}\left[\ket{\psi_n(\vec r^\prime)} \pm e^{i\left( \vec k_0 + {\bf K} \right) \cdot\vec\Delta^\prime}\ket{\psi_n(\vec r^\prime + \vec\Delta^\prime)}\right] \) are the eigenvectors of the tunnelling Hamiltonian with eigenvalues \(\pm 1\) respectively. With this basis transformation, we can rewrite eq.~\ref{greensfunction3} as
\begin{equation}\begin{aligned}
		\braket{\Psi_\text{gs} | c_{{\bf K}\sigma}(t) c^\dagger_{{\bf K}\sigma} | \Psi_\text{gs}} = \frac{1}{2N}\sum_n\sum_{\vec r,\vec r^\prime,\vec \Delta^\prime} \braket{\psi_{0}(\vec r) | c_{{\bf K}\sigma}(t) \left[ \ket{\chi_n^+(\vec r^\prime,\vec\Delta^\prime)}\bra{\chi_n^+(\vec r^\prime,\vec\Delta^\prime)} - \ket{\chi_n^-(\vec r^\prime,\vec\Delta^\prime)}\bra{\chi_n^-(\vec r^\prime,\vec\Delta^\prime)}\right] c^\dagger_{{\bf K}\sigma} | \psi_{0}(\vec r)}~.
\end{aligned}\end{equation}

In order to make the expression more transparent, we consider the various components separately:
\paragraph{\(\vec r^\prime = \vec r,~\vec\Delta^\prime=0\)}:
\begin{equation}\begin{aligned}
		\braket{\Psi_\text{gs} | c_{{\bf K}\sigma}(t) c^\dagger_{{\bf K}\sigma} | \Psi_\text{gs}} \to \frac{1}{N}\sum_n\sum_{\vec r} \braket{\psi_{0}(\vec r) | c_{{\bf K}\sigma}(t) \ket{\psi_n(\vec r)}\bra{\psi_n(\vec r)} c^\dagger_{{\bf K}\sigma} | \psi_{0}(\vec r)} ~.
\end{aligned}\end{equation}
These terms represent those contributions to the total Greens function that arise from excitations that start and end at a specific auxiliary model (at \(\vec r\)), and also evolve dynamically within the same auxiliary model. These terms are therefore exactly equal to the auxiliary model Greens function at position \(\vec r\), and are the most dominant contribution due to the localised nature of the impurity model.

\paragraph{\(\vec r^\prime \neq \vec r,~\vec\Delta^\prime=0\)}:
\begin{equation}\begin{aligned}
		\braket{\Psi_\text{gs} | c_{{\bf K}\sigma}(t) c^\dagger_{{\bf K}\sigma} | \Psi_\text{gs}} \to \frac{1}{N}\sum_n\sum_{\vec r, \vec r^\prime \neq \vec r} \braket{\psi_{0}(\vec r) | c_{{\bf K}\sigma}(t) \ket{\psi_n(\vec r^\prime)}\bra{\psi_n(\vec r^\prime)} c^\dagger_{{\bf K}\sigma} | \psi_{0}(\vec r)} ~.
\end{aligned}\end{equation}
These are more non-local contributions; they involve excitations whose time evolution is governed by a different auxiliary model than the terminal one. These contributions are highly suppressed in the Kondo screened phase because of the strong entanglement of the singlet ground state.

\paragraph{\(\vec r^\prime \neq \vec r,~\vec\Delta^\prime \neq 0\)}:
\begin{equation}\begin{aligned}
		\braket{\Psi_\text{gs} | c_{{\bf K}\sigma}(t) c^\dagger_{{\bf K}\sigma} | \Psi_\text{gs}} \to \frac{1}{2N}\sum_n\sum_{\vec r,\vec r^\prime \neq \vec r,\vec \Delta^\prime \neq 0} \braket{\psi_{0}(\vec r) | c_{{\bf K}\sigma}(t) \left[ \ket{\chi_n^+(\vec r^\prime,\vec\Delta^\prime)}\bra{\chi_n^+(\vec r^\prime,\vec\Delta^\prime)} - \ket{\chi_n^-(\vec r^\prime,\vec\Delta^\prime)}\bra{\chi_n^-(\vec r^\prime,\vec\Delta^\prime)}\right] c^\dagger_{{\bf K}\sigma} | \psi_{0}(\vec r)} ~.
\end{aligned}\end{equation}
These are the most non-local contributions; they involve excitations whose time evolution is governed by three different auxiliary models. Accordingly, these contributions are further suppressed.

We now consider each type of contribution in more detail.
\subsection{\(\vec r^\prime = \vec r,~\vec\Delta^\prime=0\)}
Restricting ourselves to just the single auxiliary model contributions gives
\begin{equation}\begin{aligned}
		\braket{\Psi_\text{gs} | c_{{\bf K}\sigma}(t) c^\dagger_{{\bf K}\sigma} | \Psi_\text{gs}} = \frac{1}{N}\sum_n\sum_{\vec r} \braket{\psi_{0}(\vec r) | c_{{\bf K}\sigma}(t) \ket{\psi_n(\vec r)}\bra{\psi_n(\vec r)} c^\dagger_{{\bf K}\sigma} | \psi_{0}(\vec r)} ~.
\end{aligned}\end{equation}
We first consider more carefully the transition operator \(\mathcal{T}_{{\bf K}\sigma} = c_{{\bf K}\sigma}\) for the 1-particle excitation giving rise to the above Greens function. Within our auxiliary model approach, gapless excitations within the lattice model are represented by gapless excitations of the impurity site, specifically those that screen the impurity site and form the local Fermi liquid. As a result, the uncoordinated \(\mathcal{T}-\)matrix for the lattice model must be replaced by a combined \(\mathcal{T}-\)matrix within the impurity model that captures those gapless excitations that occur in connection with the impurity, and projects out the uncorrelated excitations that take place even when the impurity site is decoupled from the bath.

In order to construct this auxiliary model \(\mathcal{T}-\)matrix, we note that the impurity site can have both spin and charge excitations. Considering both excitations, the modified \(\mathcal{T}-\)matrix that constructs \(k-\)space excitations in correlation with the impurity site are
\begin{equation}\begin{aligned}\label{tmatrix}
	\mathcal{T}_{{\bf K}\sigma} = c_{{\bf K}\sigma}\left(\sum_{\sigma^\prime}c^\dagger_{d\sigma} + \text{h.c.}\right) + c_{{\bf K}\sigma}\left(S_d^+ + \text{h.c.}\right)~,
\end{aligned}\end{equation}
leading to the updated expression for the complete Greens function:
\begin{equation}\begin{aligned}
	\tilde G({\bf K}\sigma; t) = -i\theta(t)\frac{1}{N}\sum_n\sum_{\vec r} \braket{\psi_{0}(\vec r) | \left[\mathcal{T}_{{\bf K}\sigma}(t) \ket{\psi_n(\vec r)}\bra{\psi_n(\vec r)} \mathcal{T}_{{\bf K}\sigma}^\dagger + \mathcal{T}^\dagger_{{\bf K}\sigma} \ket{\psi_n(\vec r)}\bra{\psi_n(\vec r)} \mathcal{T}_{{\bf K}\sigma}(t)\right] | \psi_{0}(\vec r)} ~.
\end{aligned}\end{equation}

In order to convert this into a more useful form, we use eq.~\ref{heisenberg}:
\begin{equation}\begin{aligned}
		\braket{\psi_{0}(\vec r) | \mathcal{T}_{{\bf K}\sigma}(t) | \psi_n(\vec r)} = \braket{\psi_{0}(\vec r) | e^{it H_\text{tiled} }\mathcal{T}_{{\bf K}\sigma}e^{-i t H_\text{tiled}} | \psi_n(\vec r)}~.
\end{aligned}\end{equation}
Guided by the relation in eq.~\ref{tilingPrescriptionFinal} \(H_\text{tiled} = \sum_{{\bf r}}{H}_\text{aux}({\bf r}) - N{H}_\text{cbath-nint}\) between the tiled Hamiltonian and the auxiliary model, we assume that the operator \(e^{-i t H_\text{tiled}}\) acting on the state \(\ket{\psi_n(\vec r)}\) involves the excitation energy \(E_n\) of only a single auxiliary model. Specifically, \(E_n\) is the energy of the eigenstate \(\ket{\psi_n(\vec r)}\). This is supported by the fact that this class of contributions to the Greens function is completed within a single auxiliary model. Accordingly, we replace \(H_\text{tiled}\) with \(E_n - \varepsilon_{\bf k} \), where \(\varepsilon_{\bf k}\) are the eigenenergies of the non-interacting conduction bath \({H}_\text{cbath-nint}\) and \({\bf k}\) is the crystal momentum associated with the state \(\ket{\psi_n(\vec r)}\). This momentum was found to be constrained to \({\bf k}_0 + {\bf K}\) below eq.~\ref{greensfunction3}. In the same way, the action of \(H_\text{tiled}\) on the terminal state \(\ket{\psi_{0}(\vec r)}\) gives \(E_0 - \varepsilon_{\bf k_0}\), where \(E_0\) is the ground state energy of the auxiliary model at \({\bf r}\). Applying this to our expression gives
\begin{equation}\begin{aligned}
	\tilde G({\bf K}\sigma; t) = -i\theta(t)\frac{1}{N}\sum_n\sum_{\vec r} \braket{\psi_{0}(\vec r) | \left[e^{-it\omega_p} \mathcal{T}_{{\bf K}\sigma}(t) \ket{\psi_n(\vec r)}\bra{\psi_n(\vec r)} \mathcal{T}_{{\bf K}\sigma}^\dagger + e^{-it\omega_h} \mathcal{T}^\dagger_{{\bf K}\sigma} \ket{\psi_n(\vec r)}\bra{\psi_n(\vec r)} \mathcal{T}_{{\bf K}\sigma}(t)\right] | \psi_{0}(\vec r)} ~,
\end{aligned}\end{equation}
where \(\omega_p = (E_n - \varepsilon_{\bf k_0 + \bf K}) - (E_0 - \varepsilon_{\bf k_0})\) is the particle-excitation cost and \(\omega_h = -\omega_p\) is the hole-excitation cost. We now introduce the Fourier transform \(g(\omega) = \int~dt~ e^{i\omega T}f(t)\) to obtain the frequency-domain Greens function in its spectral representation:
\begin{equation}\begin{aligned}
	\tilde G({\bf K}\sigma; \omega) = \frac{1}{N}\sum_{\vec r}\sum_n \braket{\psi_{0}(\vec r) | \left[\frac{1}{\omega - \omega_p} \mathcal{T}_{{\bf K}\sigma} \ket{\psi_n(\vec r)}\bra{\psi_n(\vec r)} \mathcal{T}_{{\bf K}\sigma}^\dagger + \frac{1}{\omega - \omega_h} \mathcal{T}^\dagger_{{\bf K}\sigma} \ket{\psi_n(\vec r)}\bra{\psi_n(\vec r)} \mathcal{T}_{{\bf K}\sigma}\right] | \psi_{0}(\vec r)} ~.
\end{aligned}\end{equation}
For each value of \({\vec r}\), the term within that summation is simply the Greens function (for the excitation \(\mathcal{T}_{{\bf K}\sigma}\)) of the auxiliary model with the impurity site at \({\vec r}\). Since all these impurity models are physically equivalent (because of translation invariance), we can replace the average over \({\bf r}\) with the value obtained from any one auxiliary model.
\begin{equation}\begin{aligned}\label{kspaceGreensFunction}
	\tilde G({\bf K}\sigma; \omega) &= \sum_n \braket{\psi_{0} | \left[\frac{1}{\omega - \omega_p} \mathcal{T}_{{\bf K}\sigma} \ket{\psi_n}\bra{\psi_n} \mathcal{T}_{{\bf K}\sigma}^\dagger + \frac{1}{\omega - \omega_h} \mathcal{T}^\dagger_{{\bf K}\sigma} \ket{\psi_n}\bra{\psi_n} \mathcal{T}_{{\bf K}\sigma}\right] | \psi_{0}} \\
									&= G^>(\mathcal{T}^\dagger_{{\bf K}\sigma}, \omega - \varepsilon_{{\bf k}_0 + {\bf K}}) + G^<(\mathcal{T}^\dagger_{{\bf K}\sigma}, \omega + \varepsilon_{{\bf k}_0 + {\bf K}})~,
\end{aligned}\end{equation}
where \(G^>(\mathcal{O}^\dagger, t) = -i\braket{\mathcal{O}(t)\mathcal{O}^\dagger}\) and \(G^<(\mathcal{O}^\dagger, t) = -i\braket{\mathcal{O}^\dagger\mathcal{O}(t)}\) are the greater and lesser Greens function for the auxiliary model.

\subsection{\(\vec r^\prime \neq \vec r,~\vec\Delta^\prime=0\)}
Allowing for more non-local contributions, and updating the \(\mathcal{T}-\)matrix similar to the previous section gives the following Greens function:
\begin{equation}\begin{aligned}
	\tilde G({\bf K}\sigma; t) = -i\theta(t)\frac{1}{N}\sum_n\sum_{\vec r, \vec r^\prime} \braket{\psi_{0}(\vec r) | \left[\mathcal{T}_{{\bf K}\sigma}(t) \ket{\psi_n(\vec r^\prime)}\bra{\psi_n(\vec r^\prime)} \mathcal{T}_{{\bf K}\sigma}^\dagger + \mathcal{T}^\dagger_{{\bf K}\sigma} \ket{\psi_n(\vec r^\prime)}\bra{\psi_n(\vec r^\prime)} \mathcal{T}_{{\bf K}\sigma}(t)\right] | \psi_{0}(\vec r)} ~.
\end{aligned}\end{equation}
In order to allow computations within a single auxiliary model, we replace the translated state \(\ket{\psi_n(\vec r^\prime)}\) with \(T^\dagger(\vec r \to \vec r^\prime)\ket{\psi_n(\vec r)} \), where \(T^\dagger(\vec r - \vec r^\prime)\) translates all sites by the vector \(\vec r^\prime - \vec r\). This leads to a modified correlation function but within a single auxiliary model:
\begin{equation}\begin{aligned}
	\tilde G({\bf K}\sigma; t) = -i\theta(t)\frac{1}{N}\sum_{n,\vec r, \vec r^\prime} \bra{\psi_{0}(\vec r)} &\left[\mathcal{T}_{{\bf K}\sigma}(t)T^\dagger(\vec r - \vec r^\prime) \ket{\psi_n(\vec r)}\bra{\psi_n(\vec r)}T(\vec r - \vec r^\prime) \mathcal{T}_{{\bf K}\sigma}^\dagger + \right. \\
								&\left. \mathcal{T}^\dagger_{{\bf K}\sigma}T^\dagger(\vec r - \vec r^\prime) \ket{\psi_n(\vec r)}\bra{\psi_n(\vec r)}T(\vec r - \vec r^\prime) \mathcal{T}_{{\bf K}\sigma}(t)\right] \ket{\psi_{0}(\vec r)} ~.
\end{aligned}\end{equation}
In fact, by the same arguments as in the previous section, we can obtain a frequency-resolved Greens function:
\begin{equation}\begin{aligned}
	\tilde G({\bf K}\sigma; \omega) = \sum_n \bra{\psi_{0}} &\left[\frac{1}{\omega - \omega_p} \mathcal{T}_{{\bf K}\sigma}(t)T^\dagger(\vec r - \vec r^\prime) \ket{\psi_n}\bra{\psi_n}T(\vec r - \vec r^\prime) \mathcal{T}_{{\bf K}\sigma}^\dagger + \right.\\
															&\left. \frac{1}{\omega - \omega_h} \mathcal{T}^\dagger_{{\bf K}\sigma}T^\dagger(\vec r - \vec r^\prime) \ket{\psi_n}\bra{\psi_n}T(\vec r - \vec r^\prime) \mathcal{T}_{{\bf K}\sigma}(t)\right] \ket{\psi_{0}} ~.
\end{aligned}\end{equation}
To obtain the above, we used the fact that \(H_\text{tiled}\) commutes with the translation operator.

%TODO
%By expanding in terms of the small parameter $\frac{\lambda_{\vec k_0}^2}{\mathcal{Z}}\frac{\mathcal{G}_{dz}}{\mathcal{G}_{dd}}$, this expression for $\Sigma_{H-H}(\vec k,\omega)$ can be recast in the following form:
%\begin{equation}\begin{aligned}
%	\Sigma_{H-H}(\vec k, \omega) \simeq \Sigma_\text{loc}(\omega) + \tilde{\Sigma}(\vec k, \omega)~,~\label{selfenergy2}
%\end{aligned}\end{equation}
%where $\Sigma_\text{loc}$ and $\Sigma_\text{near-n}$ correspond to the local and nearest-neighbour self-energies defined on the real-space lattice 
%\begin{eqnarray}
%\Sigma_\text{loc}(\omega) &=& \frac{\lambda_{\vec k_0}^2}{\mathcal{G}_{dd}^{(0)}} - \frac{\lambda_{\vec k_0}^2}{\mathcal{G}_{dd}}~,\\ 
%\tilde{\Sigma}(\vec k, \omega)&=&\xi_{\vec k}\Sigma_\text{near-n}~,~\Sigma_\text{near-n} = \left(1/G^{(0)}\right)_\text{near-n} + \frac{\lambda_{\vec k_0}^2}{\mathcal{Z}}\frac{\mathcal{G}_{dz}}{\mathcal{G}_{dd}^2}~.
%\end{eqnarray}
%We note that the form of $\Sigma(\vec k, \omega)$ obtained by us is similar to the form proposed on phenomenological grounds in the generalised DMFT + $\Sigma$ approach~\cite{sadovskii2012}.

\subsection{Equal-Time Ground State Correlators: Real-space}\label{staticCorr}
We first consider a real-space operator \(\mathcal{O}({\bf r} + {\bf \Delta})\mathcal{O}^\dagger({\bf r})\) that quantifies the presence of correlations over a distance \({\bf \Delta}\). The correlation function itself is given by the expectation value of this operator in the ground state:
\begin{equation}\begin{aligned}
	C_{\mathcal{O}}({\bf \Delta}) = \braket{\Psi_\text{gs} | \mathcal{O}({\bf r} + {\bf \Delta})\mathcal{O}^\dagger({\bf r}) | \Psi_\text{gs}}~.
\end{aligned}\end{equation}
To obtain a tractable expression for this, we first replace the full ground state with its expression in terms of the auxiliary model ground states (eq.~\ref{groundstateProposal}):
\begin{equation}\begin{aligned}
	C_{\mathcal{O}}({\bf \Delta}) = \frac{1}{N}\sum_{{\bf r}_1,{\bf r}^\prime}\braket{\psi_\text{gs}\left({\bf r}_1 + {\bf r}^\prime + {\bf r}_c + {\bf r}\right)  | \mathcal{O}({\bf r} + {\bf \Delta})\mathcal{O}^\dagger({\bf r}) | \psi_\text{gs}\left({\bf r}_1 + {\bf r}_c + {\bf r}\right)} e^{-i{\bf k}_0\cdot\left({\bf r}_1 - {\bf r}_2\right)}~,
\end{aligned}\end{equation}
where \({\bf r}_1\) and \({\bf r}_1 + {\bf r}^\prime\) are the positions of the incoming and outgoing auxiliary model states, relative to \({\bf r}_c + {\bf r}\), and \({\bf k}_0\) is the crystal momentum of the ground state (which we will immediately set to zero). In order to convert the incoming and outgoing states into the same auxiliary mode at a reference location \({\bf r}_c\) (which would then allow computations purely within a single auxiliary model), we use the relation: \(\ket{\psi_\text{gs}\left({\bf x}\right)} = T^\dagger\left({\bf x} - {\bf r}_c\right)\psi_\text{gs}\ket{\psi\left({\bf r}_c \right)}\). Substituting this appropriately for both the auxiliary model states gives:
\begin{equation}\begin{aligned}
	C_{\mathcal{O}}({\bf \Delta}) &= \frac{1}{N}\sum_{{\bf r}_1,{\bf r}^\prime}\braket{\psi_\text{gs}\left({\bf r}_c\right) | T\left({\bf r}_1 + {\bf r}^\prime + {\bf r}\right) \mathcal{O}({\bf r} + {\bf \Delta})\mathcal{O}^\dagger({\bf r}) T^\dagger\left({\bf r}_1 + {\bf r}\right) | \psi_\text{gs}\left({\bf r}_c\right)}\\
								  &= \frac{1}{N}\sum_{{\bf r}_1,{\bf r}^\prime}\braket{\psi_\text{gs}\left({\bf r}_c\right) | T\left({\bf r}^\prime\right) \mathcal{O}({\bf \Delta} - {\bf r}_1) \mathcal{O}^\dagger(- {\bf r}_1) | \psi_\text{gs}\left({\bf r}_c\right)}\\
								  &= \frac{1}{N}\sum_{{\bf r}_1,{\bf r}^\prime}\braket{\psi_\text{gs}\left({\bf r}_c + {\bf r}^\prime\right) | \mathcal{O}({\bf \Delta} - {\bf r}_1) \mathcal{O}^\dagger(- {\bf r}_1) | \psi_\text{gs}\left({\bf r}_c\right)}~.
\end{aligned}\end{equation}
As a final cosmetic change, we transform \({\bf r}_1 \to -{\bf r}\):
\begin{equation}\begin{aligned}
	C_{\mathcal{O}}({\bf \Delta}) = \frac{1}{N}\sum_{{\bf r},{\bf r}^\prime}\braket{\psi_\text{gs}\left({\bf r}_c + {\bf r}^\prime\right) | \mathcal{O}({\bf \Delta} + {\bf r}) \mathcal{O}^\dagger({\bf r}) | \psi_\text{gs}\left({\bf r}_c\right)}~.
\end{aligned}\end{equation}
Like in the Greens function calculation, we now consider the various kinds of contributions separately.

\subsubsection{Intra-auxiliary model contributions: \({\bf r}^\prime = 0\)}
These terms describe excitations that start and propagate within the same auxiliary model, upto a distance \({\bf \Delta}\):
\begin{equation}\begin{aligned}
	C_{\mathcal{O}}({\bf \Delta}) \to \frac{1}{N}\sum_{{\bf r}}\braket{\psi_\text{gs}\left({\bf r}_c\right) | \mathcal{O}({\bf \Delta} + {\bf r})\mathcal{O}^\dagger({\bf r}) | \psi_\text{gs}\left({\bf r}_c\right)}~.
\end{aligned}\end{equation}
With an eye towards introducing the impurity operators into the correlations, we insert a complete basis defined by the eigenstates \(\left\{\ket{\psi_n\left({\bf r}_c\right)}\right\} \) of the auxiliary model into the expression:
\begin{equation}\begin{aligned}
	C_{\mathcal{O}}({\bf \Delta}) \to \frac{1}{N}\sum_{{\bf r}}\sum_{n}\braket{\psi_\text{gs}\left({\bf r}_c\right) | \mathcal{O}({\bf \Delta} + {\bf r}) \ket{\psi_n\left({\bf r}_c\right)}\bra{\psi_n\left({\bf r}_c\right)} \mathcal{O}^\dagger({\bf r}) | \psi_\text{gs}\left({\bf r}_c\right)}~.
\end{aligned}\end{equation}

For the excitations that exist purely in the conduction bath \(({\bf r} \neq {\bf r}_c)\), the excitation operators must be suitably modified (see the arguments around eq.~\ref{tmatrix}) in order to incorporate Kondo screening. The modified excitation operators are generally defined as
\begin{equation}\begin{aligned}\label{operatorTmatrix}
	\mathcal{\tilde O}({\bf r}) = \mathcal{O}({\bf r})\mathcal{O}^\dagger(d) \mathcal{P}_\text{gs} ~,
\end{aligned}\end{equation}
where \(\mathcal{O}^\dagger(d)\) is the hermitian conjugate of the correlation operator \(\mathcal{O}\), but applied on the impurity sites. This therefore constitutes a time-reversed scattering process on the impurity site relative to the process in the bath. The operator \(\mathcal{P}_\text{gs}\) projects onto the ground state of the auxiliary model, since we are interested in ground state correlations. The operator multiplying the old correlation operator represents all possible excitations of the impurity site, and ensure that the bath and impurity excitation processes take place coherently. The projector ensures that only tripartite correlations between the impurity site and the two momentum states are captured by the correlation, which is what's desired within the tiling method.

With this modified operator, this class of correlation functions can be written as
\begin{equation}\begin{aligned}
	C_{\mathcal{O}}({\bf \Delta}) \to \frac{1}{N}\sum_{{\bf r} }\braket{\psi_\text{gs}\left({\bf r}_c\right) | \mathcal{\tilde O}({\bf \Delta} + {\bf r}) | \psi_\text{gs}\left({\bf r}_c\right)}\braket{\psi_\text{gs}\left({\bf r}_c\right) | \mathcal{\tilde O}^\dagger({\bf r}) | \psi_\text{gs}\left({\bf r}_c\right)}~.
\end{aligned}\end{equation}
This can be interpreted as the fact that within our auxiliary model formalism, the correlation between the sites \({\bf r}\) and \({\bf \Delta + r}\) can only occur through a transition process that connect one of the sites with the impurity site and then a return process that connects the impurity site with the other site.

\subsubsection{Inter-auxiliary model contributions: \({\bf r}^\prime \neq 0\)}
These are the most non-local contributions, and involve excitations that connect different auxiliary models:
\begin{equation}\begin{aligned}
	C_{\mathcal{O}}({\bf \Delta}) \to \frac{1}{N}\sum_{{\bf r}^\prime \neq 0}\braket{\psi_\text{gs}\left({\bf r}_c + {\bf r}^\prime\right) | \mathcal{\tilde O}({\bf \Delta} + {\bf r}) | \psi_\text{gs}\left({\bf r}_c\right)}\braket{\psi_\text{gs}\left({\bf r}_c\right) | \mathcal{\tilde O}^\dagger({\bf r}) | \psi_\text{gs}\left({\bf r}_c\right)}~.
\end{aligned}\end{equation}

\subsection{Equal-Time Ground State Correlators: Momentum-space}
We now consider momentum space correlations, through a general operator \(\mathcal{O}({\bf k}_2)\mathcal{O}^\dagger({\bf k}_1)\):
\begin{equation}\begin{aligned}
	C_{\mathcal{O}}({\bf k}_1,{\bf k}_2) = \braket{\Psi_\text{gs} | \mathcal{O}({\bf k}_2)\mathcal{O}^\dagger({\bf k}_1) | \Psi_\text{gs}}~.
\end{aligned}\end{equation}
Note that \(\mathcal{O}\) itself is a two-particle operator. To obtain a tractable expression for this, we first replace the full ground state with its expression in terms of the auxiliary model ground states (eq.~\ref{groundstateProposal}):
\begin{equation}\begin{aligned}
	C_{\mathcal{O}}({\bf k}_1,{\bf k}_2) = \frac{1}{N}\sum_{{\bf r}_1,{\bf r}_2}\braket{\psi_\text{gs}\left({\bf r}_2\right)  | \mathcal{O}({\bf k}_2)\mathcal{O}^\dagger({\bf k}_1) | \psi_\text{gs}\left({\bf r}_1\right)}~,
\end{aligned}\end{equation}
where \({\bf r}_1\) and \({\bf r}_2 \) are the positions of the incoming and outgoing auxiliary model states. In order to convert the incoming and outgoing states into the same auxiliary mode at a reference location \({\bf r}_c\) (which would then allow computations purely within a single auxiliary model), we use the relation: \(\ket{\psi_\text{gs}\left({\bf x}\right)} = T^\dagger\left({\bf x} - {\bf r}_c\right)\psi_\text{gs}\ket{\psi\left({\bf r}_c \right)}\). Substituting this appropriately for both the auxiliary model states gives:
\begin{equation}\begin{aligned}
	C_{\mathcal{O}}({\bf k}_1,{\bf k}_2) &= \frac{1}{N}\sum_{{\bf r}_1,{\bf r}_2}\braket{\psi_\text{gs}\left({\bf r}_c\right) | T\left({\bf r}_2 - {\bf r}_c\right) \mathcal{O}({\bf k}_2)\mathcal{O}^\dagger({\bf k}_1) T^\dagger\left({\bf r}_1 - {\bf r}_c\right) | \psi_\text{gs}\left({\bf r}_c\right)}~.
\end{aligned}\end{equation}
In order to simplify the translation operators, we can use eq.~\ref{translationOnField}. For that, we would need to know whether the operator \(\mathcal{O}\) involves a net transfer of momentum. If \(\mathcal{O}({\bf k})\) commutes with the total number operator \(n_{\bf k} = \sum_\sigma n_{{\bf k},\sigma}\), there is no momentum transfer. Examples of such operators are spin operators, \(S_{\alpha\beta}({\bf k}) \equiv c^\dagger_{{\bf k \alpha}}c_{{\bf k}\beta}\), and density operators \(n_{{\bf k}\alpha}\). If \(\mathcal{O}({\bf k})\) does not commute with \(n_{\bf k}\), there is a net transfer of momentum, and one such operator would be the charge isospin operator \(C^+({\bf k}) = c^\dagger_{{\bf k}\uparrow}c^\dagger_{{\bf k}\downarrow}\). For the first kind of operators, we have \(T({\bf a})\mathcal{O}({\bf k})T^\dagger({\bf a}) = \mathcal{O}({\bf k})\), while the for the latter, we get \(T({\bf a})\mathcal{O}({\bf k})T^\dagger({\bf a}) = e^{-{\bf a}\cdot{\bf 2k}}\mathcal{O}({\bf k})\), where \(2{\bf k}\) represents the momentum being transferred by the operator. We consider the two cases separately.

\subsubsection{Momentum-conserving operators: \(\left[\mathcal{O}({\bf k}), n_{{\bf k}}\right] = 0\)}
For these operators, the expression for the correlation gives
\begin{equation}\begin{aligned}
	C_{\mathcal{O}}({\bf k}_1,{\bf k}_2) &= \frac{1}{N}\sum_{{\bf r}_1,{\bf r}_2}\braket{\psi_\text{gs}\left({\bf r}_c\right) | T\left({\bf r}_2 - {\bf r}_1\right) \mathcal{O}({\bf k}_2)\mathcal{O}^\dagger({\bf k}_1) | \psi_\text{gs}\left({\bf r}_c\right)}\\
										 &=\sum_{{\bf \Delta}}\braket{\psi_\text{gs}\left({\bf r}_c + {\bf \Delta}\right) | \mathcal{O}({\bf k}_2)\mathcal{O}^\dagger({\bf k}_1) | \psi_\text{gs}\left({\bf r}_c\right)}~.
\end{aligned}\end{equation}
To obtain the last form, we defined \({\bf \Delta}={\bf r}_2 - {\bf r}_1\) as the distance between the incoming and outgoing auxiliary model states, and preformed the sum over the free variable \({\bf r}_1\) to cancel out the factor of \(1/N\). Just like before, this expression can be decomposed into a term that involves a single auxiliary model and other terms that involve two distinct auxiliary models. By making the identification of the right transition operator \(\mathcal{\tilde O}({\bf k}) = \mathcal{O}({\bf k})\mathcal{O}^\dagger(d)\mathcal{P}_\text{gs}\), we get
\begin{equation}\begin{aligned}\label{kspaceCorrelation}
C_{\mathcal{O}}({\bf k}_1,{\bf k}_2) &= \sum_{{\bf \Delta}}\braket{\psi_\text{gs}\left({\bf r}_c + {\bf \Delta}\right) | \mathcal{\tilde O}({\bf k}_2) | \psi_\text{gs}\left({\bf r}_c\right)}\braket{\psi_\text{gs}\left({\bf r}_c\right) | \mathcal{\tilde O}^\dagger({\bf k}_1) | \psi_\text{gs}\left({\bf r}_c\right)}~.
\end{aligned}\end{equation}

\subsubsection{Non-momentum-conserving operators: \(\left[\mathcal{O}({\bf k}), n_{{\bf k}}\right] \neq 0\)}
This class of operators incur an additional phase factor of \(e^{-{\bf a}\cdot{\bf 2k}}\) when the translation operators are translated across them:
\begin{equation}\begin{aligned}
	C_{\mathcal{O}}({\bf k}_1,{\bf k}_2) &= \frac{1}{N}\sum_{{\bf r}_1,{\bf r}_2}\braket{\psi_\text{gs}\left({\bf r}_c\right) | T\left({\bf r}_2 - {\bf r}_1\right) \mathcal{O}({\bf k}_2)\mathcal{O}^\dagger({\bf k}_1) | \psi_\text{gs}\left({\bf r}_c\right)}e^{2i\left( {\bf r}_1 - {\bf r}_c \right) \cdot \left( {\bf k}_1 - {\bf k}_2 \right)} \\
										 &= \delta_{{\bf k}_1, {\bf k}_2}\sum_{{\bf \Delta}}\braket{\psi_\text{gs}\left({\bf r}_c + {\bf \Delta}\right) | \mathcal{O}({\bf k}_2)\mathcal{O}^\dagger({\bf k}_1) | \psi_\text{gs}\left({\bf r}_c\right)}~.
\end{aligned}\end{equation}
For this expression, we again defined \({\bf \Delta}\) similar to before, and carried out the sum of \({\bf r}_1\) involving the exponential to obtain a factor of \(N \delta_{{\bf k}_1, {\bf k}_2}\). The Kronecker delta factor is a manifestation of translational invariance and the associated total momentum conservation.

As it stands, operators like \(\mathcal{O}({\bf k}_1)\) act purely on the conduction bath degrees of freedom. In order to incorporate impurity-bath correlation effects, we modify these operators using the appropriate \(T-\)matrices, by employing eq.~\ref{operatorTmatrix}. The final computations are carried out using these modified operators.


\subsection{Entanglement Measures}
We will now describe the prescription of calculating entanglement measures of the lattice model from within our auxiliary model treatment. In this section, we are interested mainly in two such measures, the entanglement entropy and the mutual information. Given a pure state \(\ket{\Psi}\) describing the complete system, the entanglement entropy \(S_\text{EE}(\nu)\) of a subsystem \(\nu\) quantifies the entanglement of \(\nu\) with the rest of the subsystem, and is defined as
\begin{equation}\begin{aligned}
	S_\text{EE}(\nu) = -\text{Tr}\left[\rho(\nu)\log \rho(\nu)\right],\quad \rho(\nu) = \text{Tr}_\nu\left[\ket{\Psi}\bra{\Psi}\right] 
\end{aligned}\end{equation}
where \(\text{Tr}\left[\cdot\right] \) is the trace operator, and \(\rho(\nu)\) is the reduced density matrix (RDM) for the subsystem \(\nu\) obtained by taking the partial trace \(\text{Tr}_\nu\) (over the states of \(\nu\)) of the full density matrix \(\rho = \ket{\Psi}\bra{\Psi}\). If the subsystem \(\nu\) describes local regions in real space (or states in \(k-\)space), we might be interested in the entanglement between two such subsystems \(\nu_1\) and \(\nu_2\). The correct measure to quantify such entanglement is the mutual information:
\begin{equation}\begin{aligned}
	I_2(\nu_1,\nu_2) = S_\text{EE}(\nu_1) + S_\text{EE}(\nu_2) - S_\text{EE}(\nu_1 \cup \nu_2)~,
\end{aligned}\end{equation}
where \(\nu_1 \cup \nu_2\) is a larger subsystem formed by combining \(\nu_1\) and \(\nu_2\).

\subsubsection{Real-space entanglement}
Real-space entanglement measures can be used to probe delocalisation-localisation transitions. The simplest such measure is the entanglement of a local mode. Since the local entanglement entropy will be uniform at each lattice site for a system with translation invariance, it suffices to calculate the real-space averaged entanglement entropy \(S_\text{EE}^\text{loc} = \frac{1}{N}\sum_{{\bf r}}S_\text{EE}\left( {\bf r} \right) \). By visualising the lattice model as a superposition of auxiliary models placed at various sites (eq.~\ref{tilingPrescriptionFinal}), the real space average of the lattice model can be thought of as an average over sites of a particular impurity model, and then a second average over all the impurity models. But since all all impurity models are equivalent to each other, the second average is redundant. Secondly, all correlations must derive from the impurity site, which can be formally encoded by subtracting, from this average, the corresponding contribution obtained in the absence of the impurity site. In total, the lattice-auxiliary model relation for the local entanglement entropy can be written as
\begin{equation}\begin{aligned}
	S_\text{EE}^\text{loc} = \frac{1}{N}\sum_{{\bf r}}\left[S_\text{EE}({\bf r}_d + {\bf r}) - S_\text{EE}^{(0)}({\bf r}_d + {\bf r})\right] ~,
\end{aligned}\end{equation}
where \({\bf r}_d\) is the impurity site position, \({\bf r}\) is the distance of a conduction bath site from the impurity site (can be zero), and \(S_\text{EE}({\bf r}_d + {\bf r})\) is the entanglement entropy, calculated within the impurity model, at the locatiom \({\bf r}_d + {\bf r}\). \(S^{(0)}_\text{EE}({\bf r}_d + {\bf r})\) is the same entanglement entropy, but calculated for an impurity model with vanishing impurity-bath hybridisation and bath interaction.

\subsubsection{Momentum-space entanglement}
Entanglement measures in \(k-\)space can provide valuable information regarding the Fermi surface structure and the nature of gapless excitations proximate to it. The most elementary measure is the entanglement entropy \(S_\text{EE}({\bf q})\) of a single excitation carrying momentum \({\bf q}\). Unlike \(S_\text{EE}^\text{loc}\), this does not involve an average, and involves a single computation:
\begin{equation}\begin{aligned}
	\tilde S_\text{EE}({\bf q}) = S_\text{EE}({\bf q}) - S_\text{EE}^{(0)}({\bf q})~.
\end{aligned}\end{equation}
It is possible to improve this by considering inter-auxiliary model contributions in the reduced density matrix \(\rho({\bf q})\). On the lattice model, \(\rho({\bf q})\) is defined as
\begin{equation}\begin{aligned}\label{SEE-k}
	\rho({\bf q}) = \text{Tr}_{\bf q}\left[\ket{\Psi_\text{gs}}\bra{\Psi_\text{gs}}\right] ~,
\end{aligned}\end{equation}
where \(\text{Tr}_{\bf q}\left[\cdot\right]\) is the partial trace over the Hilbert space of \({\bf q}\), and \(\ket{\Psi_\text{gs}}\) is the tiled ground state. Using eq.~\ref{groundstateProposal}, we can write the ground state in terms of those of the auxiliary model. These leads to two classes of terms, one purely within a single auxiliary model ground state \(\ket{\psi_\text{gs}}\), and the other involving transitions across auxiliary models:
\begin{equation}\begin{aligned}\label{RDMTiled}
	\rho({\bf q}) \sim \text{Tr}_{\bf q}\left[\ket{\psi_\text{gs}({\bf r}_d)}\bra{\psi_\text{gs}({\bf r}_d)}\right] + \frac{1}{N}\text{Tr}_{\bf q}\sum_{{\bf r}}\left[\ket{\psi_\text{gs}({\bf r}_d)}\bra{\psi_\text{gs}({\bf r}_d + {\bf r})}\right]~,
\end{aligned}\end{equation}
where \(\ket{\psi_\text{gs}({\bf r}_d)}\) is a reference auxiliary model ground state, and \(\ket{\psi_\text{gs}({\bf r}_d + {\bf r})}\) sums over other auxiliary models at increasing distances from this reference model. The \(\sim\) indicates that the RDM needs to be normalised. The first term leads to the expression in eq.~\ref{SEE-k}, while the second term can be used to improve this estimate. Away from any quantum critical points, the localised nature of the impurity-bath hybridisation ensures that impurity correlations decay exponentially away from the transition. This ensures that the second term is very small away from a critical point. We therefore restrict ourselves to just the first term in the present work.

\section{Properties of the Bloch states}\label{BlochProperties}

\subsection{Translation invariance}
It is easy to verify that \(\Psi_{\vec k}\left(\left\{ {\bf r}_k \right\}  \right) \) transforms like Bloch functions under translation by a displacement \({\bf r}\):
\begin{equation}\begin{aligned}
	\Psi_{\vec k}\left(\left\{{\bf r}_k + {\bf r}\right\} \right) = \frac{1}{\lambda_{\ket{\vec k}} \mathcal{Z} N}\sum_{{\bf r}_i,{\bf a}} e^{i \vec{k}\cdot\vec{R}_i} \psi_\text{aux}\left({\bf r}_i - {\bf r}, {\bf a}\right) = \frac{1}{\lambda_{\ket{\vec k}} \mathcal{Z} N}\sum_{{\bf r}_j, {\bf a}} e^{i \vec{k}\cdot\left(\vec{R}_j + {\bf r}\right)} \psi_\text{aux}\left({\bf r}_j, {\bf a}\right) = e^{i \vec{k}\cdot\vec{R}} \Psi_{\vec k}\left(\left\{ {\bf r}_k \right\} \right)
\end{aligned}\end{equation}
In the last equation, we transformed \({\bf r}_i \to {\bf r}_j = {\bf r}_i - {\bf r}\). Note that the argument \({\bf a}\) does not change under a translation of the system, because that vector always represents the difference between the impurity lattice position and its nearest neighbours, irrespective of the absolute position of the impurity.

The wavefunction can be even brought into the familiar Bloch function form:
\begin{equation}\begin{aligned}
	\Psi^n_{\vec k}\left(\left\{{\bf r}_k\right\}\right) = \sum_{{\bf r}_i, {\bf a}} \frac{e^{i \vec{k}\cdot\vec{R}_i}}{\lambda_{\ket{\vec k}} \mathcal{Z} N} \psi_\text{aux}\left(\left\{ {\bf r}_d - {\bf r}_i, {\bf r_0} - {\bf r}_i - {\bf a} \right\} \right) = \frac{e^{i \vec{k}\cdot\frac{1}{N}\sum_k \vec{r}_k}}{\lambda_{\ket{\vec k}} \mathcal{Z} N}\sum_{{\bf r}_i, {\bf a}} e^{-i \vec{k}\cdot\left(\frac{1}{N}\sum_k{\bf r}_k - \vec{R}_i\right)} \psi_\text{aux}\left(\left\{ {\bf r}_d - {\bf r}_i, {\bf r_0} - {\bf r}_i - {\bf a} \right\} \right) \\
	= e^{i \vec{k}\cdot\vec{r}_\text{COM}} \eta_{\vec k}\left(\left\{{\bf r}_k\right\}\right)
\end{aligned}\end{equation}
where \({\bf r}_\text{COM} = \frac{1}{N}\sum_k {\bf r}_k\) is the center-of-mass coordinate and \(\eta_{\vec k}\left(\left\{{\bf r}_k\right\}\right) = \frac{1}{\lambda_{\ket{\vec k}} \mathcal{Z} N}\sum_{{\bf r}_i} e^{-i \vec{k}\cdot\left(\frac{1}{N}\sum_k{\bf r}_k - \vec{R}_i\right)} \psi^n_\text{aux}\left(\left\{{\bf r}_k - {\bf r}_i\right\}\right)\) is the translation symmetric function. This form of the eigenstate allows the interpretation that tuning the Bloch momentum \(\vec k\) corresponds to a translation of the center of mass of the system (or in this case, of the auxiliary models that comprise the system).

\subsection{Orthonormality}
It is straightforward to show that these states form an orthonormal basis. We start by writing down the inner product of two distinct such states:
\begin{equation}\begin{aligned}
	\braket{\Psi_{\vec k^\prime} | \Psi_{\vec k}} 
	&= \frac{1}{\lambda_{\ket{\vec k^\prime}}^* \lambda_{\ket{\vec k}} \mathcal{Z}^2 N^2}\sum_{{\bf r}_i, {\bf r}_j, {\bf a}, {\bf a}^\prime} e^{i \left(\vec{k}\cdot\vec{R}_i - \vec{k}^\prime\cdot\vec{R}_j\right) } \braket{\psi_\text{aux}\left( {\bf r}_j, {\bf a}^\prime \right) | \psi_\text{aux}\left( {\bf r}_i, {\bf a} \right)} \\
	&= \frac{1}{\lambda_{\ket{\vec k^\prime}}^* \lambda_{\ket{\vec k}} \mathcal{Z}^2 N^2}\sum_{{\bf r}_i, {\bf r}_j, {\bf a}, {\bf a}^\prime} e^{i \left(\vec{k}\cdot\vec{R}_i - \vec{k}^\prime\cdot\vec{R}_j\right) } \braket{\psi_\text{aux}\left( {\bf r_0}, {\bf a}^\prime \right) T^\dagger\left( {\bf r_0} - {\bf r}_j \right) T\left( {\bf r_0} - {\bf r}_i \right) | \psi_\text{aux}\left( {\bf r_0}, {\bf a} \right)}\\
\end{aligned}\end{equation}
At this point, we insert a complete basis of momentum eigenkets \(1 = \sum_{\vec q}\ket{\vec q}\bra{\vec q}\) to resolve the translation operators:
\begin{equation}\begin{aligned}
	\braket{\Psi_{\vec k^\prime} | \Psi_{\vec k}} &= \frac{1}{\lambda_{\ket{\vec k^\prime}}^* \lambda_{\ket{\vec k}} \mathcal{Z}^2 N^2}\sum_{{\bf r}_i, {\bf r}_j, {\bf a}, {\bf a}^\prime, \vec q} e^{i \left(\vec{k}\cdot\vec{R}_i - \vec{k}^\prime\cdot\vec{R}_j\right) } \braket{\psi_\text{aux}\left( {\bf r_0}, {\bf a}^\prime \right) T\left( {\bf r}_j - {\bf r}_i \right) \ket{\vec q}\bra{\vec q} \psi_\text{aux}\left( {\bf r_0}, {\bf a} \right)}\\
						      &= \frac{1}{\lambda_{\ket{\vec k^\prime}}^* \lambda_{\ket{\vec k}} \mathcal{Z}^2 N^2}\sum_{{\bf r}_i, {\bf r}_j, {\bf a}, {\bf a}^\prime, \vec q} e^{i \left(\vec{k}\cdot\vec{R}_i - \vec{k}^\prime\cdot\vec{R}_j\right) } \braket{\psi_\text{aux}\left( {\bf r_0}, {\bf a}^\prime \right) e^{i \vec q \cdot \left( {\bf r}_j - {\bf r}_i \right)} \ket{\vec q}\bra{\vec q} \psi_\text{aux}\left( {\bf r_0}, {\bf a} \right)} \\
						      &= \frac{1}{|\lambda_{\ket{\vec k}}|^2 \mathcal{Z}^2} \delta_{\vec k, \vec k^\prime}\sum_{{\bf a}, {\bf a}^\prime}\braket{\psi_\text{aux}\left( {\bf r_0}, {\bf a}^\prime \right) \ket{\vec k}\bra{\vec k} \psi_\text{aux}\left( {\bf r_0}, {\bf a} \right)}\\
\end{aligned}\end{equation}
At the last step, we used the summation form of the Kronecker delta function: \(\sum_{{\bf r}_i}e^{i {\bf r}_i \left( \vec k - \vec q \right) }\sum_{{\bf r}_j}e^{i {\bf r}_j \left( \vec q - \vec k^\prime \right) } = N \delta_{\vec k,\vec q}N \delta_{\vec k^\prime,\vec q}\). The final remaining step is to identify that the inner products inside the summation are actually independent of the direction \({\bf a}, {\bf a}^\prime\) of the connecting vector, and both are equal to the normalisation factor \(\lambda_{\ket{\vec k}}\):
\begin{equation}\begin{aligned}
	\braket{\Psi_{\vec k^\prime} | \Psi_{\vec k}} = \frac{1}{|\lambda_{\ket{\vec k}}|^2 \mathcal{Z}^2 } \delta_{\vec k, \vec k^\prime}\sum_{{\bf a},{\bf a}^\prime} |\lambda_{\ket{\vec k}}|^2 = \frac{1}{|\lambda_{\ket{\vec k}}|^2 \mathcal{Z}^2 } \delta_{\vec k, \vec k^\prime}w^2 |\lambda_{\ket{\vec k}}|^2 = \delta_{\vec k, \vec k^\prime}
\end{aligned}\end{equation}
This concludes the proof of orthonormality.

\subsection{Eigenstates}
To demonstrate that these are indeed eigenstates of the bulk Hamiltonian, we will calculate the matrix elements of the full Hamiltonian between these states:
\begin{equation}\begin{aligned}
	&\bra{\Psi_{\vec k^\prime}}H_\text{H-H} \ket{\Psi_{\vec k}} = \frac{1}{\lambda^*_{\ket{\vec k^\prime}} \lambda_{\ket{\vec k}} \mathcal{Z}^2 N^2}\sum_{{\bf r}_i, {\bf r}_j, {\bf r}_k, {\bf a}, {\bf a}^\prime}e^{i\left(\vec{k}\cdot\vec{R}_i - \vec{k}^\prime\cdot\vec{R}_j\right)} \bra{\psi_\text{aux}\left({\bf r}_j, {\bf a}^\prime\right)} H_\text{aux}\left({\bf r}_k\right) \ket{\psi_\text{aux}\left({\bf r}_i, {\bf a}\right)}\\
	&= \frac{1}{\lambda^*_{\ket{\vec k^\prime}} \lambda_{\ket{\vec k}} \mathcal{Z}^2 N^2}\sum_{{\bf r}_i, {\bf r}_j, {\bf r}_k, {\bf a}, {\bf a}^\prime}e^{i\left(\vec{k}\cdot\vec{R}_i - \vec{k}^\prime\cdot\vec{R}_j\right)} \bra{\psi_\text{aux}\left({\bf r_0}, {\bf a}^\prime\right)} T^\dagger_{{\bf r_0} - {\bf r}_j} T_{{\bf r_0} - {\bf r}_k} H_\text{aux}\left({\bf r_0}\right) T^\dagger_{{\bf r_0} - {\bf r}_k} T_{{\bf r_0} - {\bf r}_i} \ket{\psi_\text{aux}\left({\bf r_0}, {\bf a}\right)}\\
	&= \frac{1}{\lambda^*_{\ket{\vec k^\prime}} \lambda_{\ket{\vec k}} \mathcal{Z}^2 N^2}\sum_{{\bf r}_i, {\bf r}_j, {\bf r}_k, {\bf a}, {\bf a}^\prime}e^{i\left(\vec{k}\cdot\vec{R}_i - \vec{k}^\prime\cdot\vec{R}_j\right)} \bra{\psi_\text{aux}\left({\bf r_0}, {\bf a}^\prime\right)} T^\dagger_{{\bf r}_k - {\bf r}_j} H_\text{aux}\left({\bf r_0}\right) T_{{\bf r}_k - {\bf r}_i} \ket{\psi_\text{aux}\left({\bf r_0}, {\bf a}\right)}
\end{aligned}\end{equation}
To resolve the translation operators, we will now insert the momentum eigenstates \(\ket{\vec k}\) of the full model: \(1 = \sum_{\vec q}\ket{\vec q}\bra{\vec q}\) in between the operators.
\begin{equation}\begin{aligned}
	\bra{\Psi_{\vec k^\prime}}H_\text{H-H} \ket{\Psi_{\vec k}} &= \sum_{{\bf r}_i, {\bf r}_j, {\bf r}_k, \atop{{\bf a}, {\bf a}^\prime, \vec q, \vec q^\prime}}\frac{e^{i\left(\vec{k}\cdot\vec{R}_i - \vec{k}^\prime\cdot\vec{R}_j\right)}}{\lambda^*_{\ket{\vec k^\prime}} \lambda_{\ket{\vec k}} \mathcal{Z}^2 N^2} \braket{\psi_\text{aux}\left({\bf r_0}, {\bf a}^\prime\right)| \vec q^\prime}\bra{\vec q^\prime} T^\dagger_{{\bf r}_k - {\bf r}_j} H_\text{aux}\left({\bf r_0}\right) T_{{\bf r}_k - {\bf r}_i}  \ket{\vec q}\braket{\vec q | \psi_\text{aux}\left({\bf r_0}, {\bf a}\right)}\\
								   &= \sum_{{\bf r}_i, {\bf r}_j, {\bf r}_k, \atop{{\bf a}, {\bf a}^\prime, \vec q, \vec q^\prime}}\frac{e^{i\left(\vec{k}\cdot\vec{R}_i - \vec{k}^\prime\cdot\vec{R}_j\right)}}{\lambda^*_{\ket{\vec k^\prime}} \lambda_{\ket{\vec k}} \mathcal{Z}^2 N^2} \braket{\psi_\text{aux}\left({\bf r_0}, {\bf a}^\prime\right)| \vec q^\prime} e^{-i \vec q^\prime \cdot \left({\bf r}_k - {\bf r}_j\right) } e^{i \vec q \cdot \left({\bf r}_k - {\bf r}_i\right)} \braket{\vec q^\prime | H_\text{aux}\left({\bf r_0}\right) | \vec q}  \braket{\vec q | \psi_\text{aux}\left({\bf r_0}, {\bf a}\right)}\\
								   &= \frac{N}{\lambda^*_{\ket{\vec k^\prime}} \lambda_{\ket{\vec k}} \mathcal{Z}^2}\sum_{{\bf a}, {\bf a}^\prime, \vec q, \vec q^\prime} \delta_{\vec k, \vec q} \delta_{\vec q,\vec q^\prime} \delta_{\vec q^\prime,\vec k} \braket{\psi_\text{aux}\left({\bf r_0}, {\bf a}^\prime\right)| \vec q^\prime} \braket{\vec q^\prime | H_\text{aux}\left({\bf r_0}\right) | \vec q}  \braket{\vec q | \psi_\text{aux}\left({\bf r_0}, {\bf a}\right)}\\
								   &= \frac{N}{|\lambda_{\ket{\vec k}}|^2 \mathcal{Z}^2}\delta_{kk^\prime}\sum_{{\bf a}, {\bf a}^\prime} \braket{\psi_\text{aux}\left({\bf r_0}, {\bf a}^\prime\right)| \vec k} \braket{\vec k | H_\text{aux}\left({\bf r_0}\right) | \vec k}  \braket{\vec k | \psi_\text{aux}\left({\bf r_0}, {\bf a}\right)}\\
								   &= \frac{N}{|\lambda_{\ket{\vec k}}|^2 \mathcal{Z}^2}\delta_{kk^\prime}\braket{\vec k | H_\text{aux}\left({\bf r_0}\right) | \vec k} \underbrace{\sum_{{\bf a}, {\bf a}^\prime} \braket{\psi_\text{aux}\left({\bf r_0}, {\bf a}^\prime\right)| \vec k} \braket{\vec k | \psi_\text{aux}\left({\bf r_0}, {\bf a}\right)}}_{w^2 |\lambda_{\ket{\vec k}}|^2}\\
	&= N\delta_{kk^\prime}\braket{\vec k | H_\text{aux}\left({\bf r_0}\right) | \vec k}
\end{aligned}\end{equation}
The momentum eigenket \(\ket{\vec k}\) can be expressed as a sum over position eigenkets at varying distances from \({\bf r_0}\). This form allows a systematic method of improving the energy eigenvalue estimate, because the interacting in the impurity model is extremely localised, so the overlap between two auxiliary models decreases rapidly with distance. The majority of the contribution will come from the state at \({\bf r_0}\), and improvements are then made by considering auxiliary models at further distances. 

